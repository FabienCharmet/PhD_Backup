In this section we take a closer look at the security of the live migration of VMs.
We first detail the existing vulnerabilities and attacks possible and then detail different security solutions, either using the migration process as a way to overcome security issues or that analyze and improve the security of the migration process itself.

\subsubsection{Attacking the VM migration process}
Traditional hypervisors have been around for decades now, and the live migration of a virtual machine is a well known process. There has been several security surveys on the topic of VM hypervisors~\cite{Reuben2007,Rehman2013,Sahoo2010,Perez-Botero2013}, and on the security of Cloud environments~\cite{cloudenvironmentsecuritysurvey-fernandes2014}.
Most of the security issues with VMs are related to the unlawful interactions between a VM and other VMs or the hypervisor.
An attacker may exploit vulnerabilities in the hypervisor and force it to execute arbitrary code or access restricted memory zone.
Another type of attack consists in corrupting the resource scheduler of the hypervisor, thus preventing other users to get the allocated resources they deserve.
We will focus here on the attacks that can impact the (live) migration of a VM.
Oberheide~\etal proposed in~\cite{empirical-oberheide2008} several attack classes that will affect the migration of VMs: attacks on the control plane, the data plane and the migration module.
First of all, attacks impacting the control plane can be summarized into two categories. The attacker is triggering unlawful migration either to move a VM toward an unsecured physical location or to overload the local hypervisor and preventing a legitimate user to be served with his due resources.
The second attack consists in Fake Resource Advertising. This kind of attacks are particularly efficient in systems where the migration is automated, because the generation of fake servers inside the topology can make the hypervisor trigger migration of VM on resources that do not exist.
Attacking the data plane during the live migration can take two forms: passively exfiltrating sensitive information or actively altering the traffic during the migration.
In the first case, the attack result in a loss of confidentiality for the stolen data and it might be exploited later on to launch other attacks on the VM.
In the second case, altering the traffic can either create a Denial-of-Service because the VM virtual disk has been compromised, or in a potentially even more harmful scenario the VM keeps working but under a compromised state, as the ongoing processes may not function with expected data.
The VM migration process can also be attacked to allow an attacker to have his VM share the same physical server with his victim's VM. Stalling the migration is studied in~\cite{stalling-atya2017} where the attacker prevents his and his victim's VMs from being separated so he can completely invalidate the memory pages of the victim's VM. It is shown that maintaining the co-residency between the VMs allows an attacker to use side channels attacks that will have a bigger impact on the victim.
A presentation on these side channel attacks is shown in~\cite{malicious-atya2017}.

\subsubsection{Protecting the VM migration process}
The Moving Target Defense (MTD) paradigm is a common technique to limit the co-residency rate between an attacker and its victim, thus leading to a reduced attack surface for side channel attacks.
Zhang~\etal introduce a Game Theory model~\cite{incentivemtd-Zhang2012} to represent the incentives a Cloud Provider has to migrate VMs in his infrastructure versus an attacker who will benefit from the co-residency. Results show that it is not necessary to permanently migrate all VMs and only a subset of them. In addition to this, the game can be extended to include cooperation between the legitimate users and the Cloud Provider.
A similar approach using MTD against side channel attacks is presented with Nomad~\cite{nomad-Moon2015b}. Nomad is built on the assumptions that the attacker cannot be precisely located inside the infrastructure, can launch potentially unknown side channels attacks and can precisely target his victim. 
The migration process proposed in Nomad provides tenants with an API to describe a set of constraints on their VM that has to be respected throughout the migration.
Nomad provides a scalable VM placement algorithm, a framework that does not require to modify the OS, the hardware or the hypervisor, other than the migration algorithms.
Protecting the migration process by making undetectable by an attacker is studied in~\cite{stealth-Achleitner2017a}. Even if encryption is applied to the migration traffic, an attacker located inside infrastructure still may distinguish migration traffic from the legitimate one. Indeed, migration traffic can easily be characterized by as set of data bursts through the network. The solution proposed uses a token bucket traffic generator to implement a white noise that will dynamically adapt to the traffic generated by the migration. This method will make each traffic look like a constant flow of indistinguishable data.

\subsubsection{Summary}
We have presented a broad overview of the Live VM migration process.
Legacy hypervisors have extensively studied, and the use of the migration process has been used not only for maintenance purposes but to tackle security issues existing in virtualization infrastructure.
The security of the migration process has also been studied and outlined vulnerabilities existing in hypervisors.
We note that the migration process extends the attack surface of the hypervisor because it also relies on the network infrastructure to support the operations of the VM migration.
Additionally, the migration of VM across SDN networks has obtained some focus~\cite{Datacenters2014,Lin2013,Ibn-Khedher2015} and a security study should be considered as well.