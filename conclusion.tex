
\label{sec:thesis_conclusion}
Network virtualization offers new service possibilities, in which a tenant can define his own network to interconnect Virtual Machines to operate his business. The tenant may request specific resources or define constraints for his Virtual Network, \eg minimum bandwidth, geographical location of the physical substrate, security requirements for the network traffic, \etc
Virtual Network migration is a process used by an infrastructure provider to ensure a certain level of service availability for the user.
This common maintenance operation may occur due to a physical failure of a network equipment or because the network is under an ongoing attack.

Existing literature presents an extensive security study of Virtual Machines and of the related migration process, evaluating existing migration solutions and considering different types of attackers. We advocated that the virtualization of the network resource is subject to a similar attack surface because of the nature of the virtualization mechanism. This similarity still stands for the migration process itself, and despite the scarce literature on the topic, consequences on data security violations remain an important matter (\eg data breaches of personal information).

In this thesis, we investigated the security of the migration of a Virtual Network in an SDN environment. 
We sought to answer three questions regarding this matter:

\begin{itemize}
    \item How can the security of the migration be characterized ?
    \item How can the security of the migration be verified ?
    \item How to optimize the monitoring of the migration process? 
\end{itemize}

We focused on a formal approach, providing a model to express security properties, how to monitor them in a real infrastructure, and how to formally verify their validity. 
We covered a broad attack surface, considering the data impacted by the migration as well as the underlying physical infrastructure that supports the virtualization operations and the migration process. 
This approach is based on the assumption that it is possible to detect every attack and monitor every network activity in the infrastructure. 
We proposed to alleviate this assumption to enable the model to be used in a realistic environment, and we formulated a resource allocation problem to do so.

The first contribution presents a formal model to characterize the security of the migration process. We considered several security properties and described them using a temporal first order logic. Traditional properties like Confidentiality, Integrity and Availability are modeled in the context of an SDN infrastructure.
% , and we presented a novel property to model the colocation ratio between Virtual Network users: the co-residency. 
% Co-residency is a security metric we recommend should be monitored to restrict the attack surface between two end users.
We implemented our model in LISP, using the theorem prover SNARK.
The proof computed with SNARK can be used to determine which event caused the security violation and when it happened.
% We considered an attacker capable of rendering part of the network infrastructure unavailable and capable of accessing or modifying sensitive information.
% Results show that the attack is detected by the theorem prover and when it occured.

Related work on formal languages for network security provides several models of networks and interactions between different actors in a communication. The lack of models for network virtualization, the Virtual Network migration process and the related security has been filled with our model. 

The second contribution investigates a Markov Decision Process to determine the optimal set of network nodes to monitor the infrastructure. This formalism is efficient for planning and decision making problem.

% Game Theory considers the interactions between several players, here a defender and an attacker.
% The attacker attacks the migration process under a constraint budget. Similarly, the defender can either use deception or monitoring techniques to protect the infrastructure and the migration.
% The different types of games explored have proven to be inadequate in their formalism or resolution to accurately represent and solve the resource allocation problem.

A Markov Decision Process considers the interactions between an agent and a system, here the defender and its infrastructure. Markov Decision Processes have been used for resource allocation problems in the literature, but their use for security purposes remains limited. We modeled the evolution of the system during the migration, while the defender will deploy monitoring resources on network nodes and will be rewarded based on the performance of the detection.
% giving the possibility to the defender to either deploy or remove monitoring resources from a node at a time throughout the migration. 
Precisely, rewards are determined by computing the impact of attacks on the migration and evaluating how many nodes are actively participating in the detection of the current attack.
% We considered a threat model similar to the one presented in Section~\ref{sec:formal_model}, and we implemented attacks with the use of the transitions between states, depending on the likelihood of a node being the target of the next attack. 
Results show that the type of physical topology and the number of attackers are the main criteria impacting the allocation of monitoring resources, while the probability of detecting an attack was only playing a minor part in the process. 
Our contribution illustrates the capacities of MDPs to simulate the behaviour of an attacker without having to formally integrate him as an extra agent in the model.
We also adapted the dynamic solution provided by the MDP into an \textit{a priori} deployment strategy of monitoring resources. This strategy is more realistic as it consists in deploying security resources prior to any migration instead of adapting the allocation every time a migration occurs.
% , comparatively to the game we proposed.

The focus of our study was centered around the physical infrastructure and the embedding of Virtual Networks on top of it.
% We have illustrated the exploitation of the migration process using several existing network attacks.
There are two main security aspects that have been left out of this study, namely the interactions between the tenant and the hypervisor, as well as the security of the internal components of the network hypervisor. These aspects are not related to network security, but they belong to the fields of cryptography and secure development.

Additionally, the focus of the resource allocation problem was set on the defender's point of view, by transferring the impact of the attacker's capacities on transitions and rewards. This aspect may not be suited for other types of attacks that could be more complex, for instance when the attack on one node is multi-staged.

The migration of Virtual Networks was a challenging research topic from a security perspective, because of the recent advances in networking techniques and the lack of specific formalisms and related solutions. We proposed a model in this thesis that sought to be the first step towards advanced evaluation models. We also focused on the applicability of this model and problem resolution tools to be usable in a real life environment. From this starting point, we envision several research leads that can be explored to complete this work on different aspects.

\textbf{Perspectives for future work}

We outline several research perspectives among different topics: 

\paragraph{Extending the formal model}
\begin{itemize}
    \item 
    The model we have developed in this thesis describes the traditional security properties considered in a computer system. We adapted the co-residency between Virtual Machines to propose a definition of the co-residency of Virtual Networks, based on the physical substrate they share. This property can be exploited by monitoring the number of migrations happening in the infrastructure, and determine which end users' networks are regularly migrated. Co-residency can be defined as a threshold to prevent two Virtual Networks from excessively sharing the same physical substrate. This has been previously highlighted as a potential sign of a malicious behaviour. Characterizing co-residency through statistical analysis of user requests and migration events allow to model user profiles that can be exploited for security purposes. For instance, the network hypervisor can implement Virtual Network migration to reduce the co-residency between users, thus limiting the existing attack surface~\cite{nomad-Moon2015b,malicious-atya2017}.
% , similarly to the works presented in~\cite{nomad-Moon2015b,malicious-atya2017}.

    \item 
    Live migration of Virtual Machines is a well studied topic, and the characterization of the network behaviour during the migration has been considered from a security point of view.
    The VM migration process has been characterized by a statistical distribution of the bandwidth usage~\cite{stealth-Achleitner2017a}.
    We suggest to apply a similar method where the quantity of flow rules, deployed in the network subsequent to the migration, is monitored and statistically modeled. This approach can serve as a basis for the definition of a stealth property~\cite{stealth-Achleitner2017a} related to the network migration process. This definition can be exploited to design a traffic noise generator that would minimize the detectability of the whole process.

% The migration traffic can be combined with noise to make the migration process undetectable. Investigating the behavior of network nodes as well as characterizing the network traffic with a statistical model may serve as a basis for the definition of a stealth property related to the network migration process. 

    \item 
    Regarding the characterization of the security properties, implementing secure-by-design migration primitives becomes the second step toward a secured migration process. These primitives describe how the Virtual Network configuration is deployed inside the physical infrastructure, and propose specific metrics and probes to be used in the generation of the execution trace. Obvious cryptographic protocols will partly answer the confidentiality aspect but these may not be applicable everywhere and other properties (\eg availability, co-residency) are not fit for this solution. We envision a migration scheduling algorithm as a counter-measure to the information collection capacities of the attacker considered in the threat model of this thesis. Based on the supposed location of the attacker in the infrastructure, configuration rules may be routed in the infrastructure while avoiding being in the attacker's range. Additionally, the destination substrate may be computed to avoid an overlap with the attacker's alleged compromised nodes.
    
\end{itemize}

\paragraph{Extending the MDP model}
\begin{itemize}
    \item 
    The resource allocation problem was modeled to consider the migration of only one Virtual Network at a time. While this assumption holds because the attacker is targeting a single user, it is weakened when several Virtual Networks are migrated simultaneously. Introducing uncertainty about the potential targets in the algorithm makes it more difficult to determine the optimal monitoring nodes. A solution to this problem is to construct a graph representing the embedding of each Virtual Network, and to remove tenants from the graph when an attack has not impacted the substrate of his Virtual Network. Pruning and weighting techniques are evaluated in~\cite{pruning-secu}.

    \item 
    As we have previously highlighted, the attack surface considered in this thesis focused on preventing unlawful manipulation or access to sensitive traffic or network equipments. Some network hypervisors focused on resource isolation, to maintain a fair use between users. A potential way to compromise the migration process is to throttle it by attacking the physical network (data plane) and/or the communication channel between tenants and the hypervisor (control plane). We propose to secure the control plane by implementing a resource allocation scheme to protect the availability of the communication channel between tenants and the hypervisor. For instance, a tenant flooding the communication channel could be timed out to and his packets to the hypervisor would be dropped at the network level. This would allow other tenants to interact with the hypervisor. We suggest to secure the data plane by designing an adaptive migration algorithm that can recompute the VN embedding as the migration goes on in case the destination substrate has been compromised during the migration. 

    \item 
    The scalability issues incurred by the large number of states generated to compute the optimal policy of a Markov Decision Process represents a computationally heavy problem for the implementation of our work inside a Cloud infrastructure composed of thousands of network nodes, Virtual Machines and tenants. We may investigate this aspect by proposing the clustering of the infrastructure into small kernels and apply our MDP resolution on each kernel. Then, we would propose a combining technique that would tie kernels together. A similar idea is investigated in~\cite{POMDP-clustering}.

    \item 
    We may consider Partially Observable Markov Decision Processes (POMDP) to extend the impact of network attacks.
    The formalism used in this thesis supposes that the agent is immediately aware of which node has been compromised by the attacker. The use of a POMDP introduces uncertainty in the decision process due to the inability of the agent to accurately determine which node has been compromised on a real-time basis. POMDP would also imply to explore new directions for the definition of the \textit{a priori} deployment. Instead of considering the performance of the monitoring set, the \textit{a priori} with a POMDP should also consider the topology of the infrastructure to determine if a node's location makes it more relevant to monitor. Attack detection using POMDP has been explored in~\cite{RL-POMDP}.
    
    \item 
    The modeling of the MDP and the different components supposed a uniformity in the detection capacity and behaviour of elements in the infrastructure (\eg attacks have identical execution traces independently from the attack source or the underlying network equipments). Evaluating the behaviour of physical network equipments under realistic attack scenarios allows to categorize them (\eg average response time, CPU consumption). This categorization can be then leveraged to augment the representation of compromised nodes during the migration. In our model, network nodes are trusted by the hypervisor. Compromised nodes may not provide the hypervisor with their actual flow tables or performance metrics to hide their malicious behaviour. Nodes may also be categorized based on a statistical analysis of their communication with the other nodes and the hypervisor (\eg types of messages exchanged and frequency). This analysis should be performed under normal circumstances and when the node is under attack.
    
\end{itemize}
% We improved the design of our verification model for its use in a realistic context by considering its deployment under an imperfect and resource constrained environment. 

% More precisely, we have proposed a formal representation of the security, mapped this model with concrete networking events and generated a formal proof of the verification of the security. 

% Finally, we designed a resource allocation problem to alleviate the hypothesis made for the formal model that network monitoring was perfect and cost free. We studied two formalism to solve this problem, namely Game Theory and Markov Decision Processes.  
% \FC{Leveraging security techniques to determine a better substrate for the migration during an attack }
% \FC{Check reference architecture -> security component ideas}
% \FC{Extending model for multi user topology etc.}
% \FC{Verifying specific predicates}
% \FC{dynam}


To conclude, the topic of our research focused on two different aspects of network security.
The first aspect was the modeling of the security problem while the second aspect was about optimizing detection mechanisms in the model. It has been highlighted in the literature that a common limitation in formal models is their excessive simplification of the real system to be computationally solvable. This leads to the solution of a problem that cannot be applied in real life anymore. The optimization scheme we proposed is based on a Markov Decision Process. This formalism has rarely been used in a security context, especially for Resource Allocation problems. 
This allowed us to partially alleviate the impact on the realism of our model. However, it remains complex to evaluate the realism of a model when no standard nor guidelines defining common basis have been published yet. We advocate that for a formal approach to be realistic, it should be designed with solutions to compensate assumptions made about the real system. Additionally, the model should be evaluated against a set of real life metrics that represent the system working under normal conditions.
% \GB{TBC}

% To conclude, the topic of our research was focused on a specific aspect of network virtualization, but we strongly believe that the methodology used throughout this thesis should be applied to a wider domain. We suggest to investigate the topic of social engineering, and precisely lateral movement attacks. We consider the scenario of an attacker collecting sensitive information in an organisation. The attacker uses phishing attacks to characterize the behavior of his targets. The behavior includes average response time and types of information collected during the phishing phase. Then, he may construct scenarios with MDPs to help him determine which target should be attacked using which methodology. 
% \CK{C'est tres eloigne de la virtualisation reseau pour le coup, c'est assez WTF. Par ailleurs, on se doute bien que les MDPs peuvent modeliser d'autres problemes. Pourquoi pas une conclusion generale sur la maniere dont tu vois l'agencement entre modeles formels et mathematiques pour traiter un probleme de securite reseau ? Est-ce que cette combinaison constitue une approche nouvelle ? Est-elle suffisante ou faudrait-il lui adjoindre quelque chose de supplementaire pour etre plus robuste ?}
% A real challenge in this approach lies in accurately modeling the expected behavior of the targets as well as collecting enough data to generate a realistic set of states for the MDP.