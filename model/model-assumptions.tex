We make several assumptions related to the scope of our study and about certain technical specificities.

\subsubsection{Predicates}
We consider in our formal model that the execution trace it analyzes is complete.
This implies that all the information necessary to compute the proof is present in the trace and that there is no missing information.
This assumption holds in our context because we have implemented a simulator to evaluate our use case. However, applying the formal model on experimental prototypes or commercial solutions will raise issues we will treat in the next chapter.

We consider that all the predicates related to determining whether a user is allowed to access a specific node or data are enforced using traditional Access Control solutions.
There is no significant contribution to be made on this aspect.

When a user accesses a node or a data, we do not differentiate the type of access being made, whether it was physical or remotely done.
Similar to the previous point, Access Control solutions are perfectly fit for these questions.

\subsubsection{Infrastructure}
We consider that the physical links in the network infrastructure are safe.
This suppose that on a practical point of view, congestion issues and man-in-the-middle attacks will only be considered on physical nodes in the infrastructure.

\subsubsection{Virtualization}
The implementation of the virtualization layers depends on the solution considered.
Each solution comes with its own identifiers, flow space and limitations on how the user may design its virtual network and with how much granularity he may implement flow rules in its VN. Therefore, we exclude from our formal model the consideration of the technical specificities of the network hypervisor.
We justify this choice that ultimately, the hypervisor will generate a set of flow rules that will be installed in the infrastructure and therefore these rules can be seen as the final representation of the formal predicates describing the embedding of the VN.

Certain migration works and survivable VNE solutions consider path splitting to provide robustness to the infrastructure.
We exclude this specificity from the model because from a formal point of view and limit the embedding of a virtual link to a single physical set of nodes.
