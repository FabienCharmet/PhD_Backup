\label{sec:security_prop}
Modeling the SDN virtualization environment is a first step towards the study of the migration process. The second step is to define which aspects of the process we need to observe.

In this section, we define the different security properties we consider during the migration of virtual networks.
We propose to model several security properties: confidentiality, integrity, availability, usually referred to as CIA~\cite{ISO/IEC270012013}. We also introduce co-residency, a property related to the virtualization process. This property represents how much two users share a common set of physical resources for the embedding of their Virtual Network.
After providing a formal definition of each security property, we refine it into several properties related to the operation of SDN networks and virtualization. 
Each property impacts physical elements of the SDN infrastructure, and we provide similar properties for the virtual nodes and links.

\subsection{Confidentiality}
\label{sec:prop-conf}
Confidentiality is a characteristic related to information.
To protect and preserve the confidentiality of information means to ensure that it is not made available or disclosed to unauthorized entities~\cite{ISO/IEC270012013}.
Therefore, in network security, the confidentiality of a message in transit in the network is defined by the ability of an authorized  user to read data forwarded by an element of the network.
Formally we can define a predicate $confidentiality(D)$ as follows:

\begin{myformula}
$ confidentiality(D) \Leftrightarrow \forall U,t~reads(U,D,t) \Rightarrow is\_data\_authorized(U,D,t)$
\end{myformula}

\begin{itemize}
\item $reads(U,D,t)$ is true if user $U$ reads data $D$ at time $t$.
\item $is\_data\_authorized(U,D,t)$ is true if user $U$ has the right to read data $D$ at time $t$.
%\item $\overline{reads(U,D,t)}$ is the negation of $reads(U,D,t)$
\end{itemize}
Simply put, if at any time no user reads $D$ or every user reading $D$ is authorized to do so, then the confidentiality of $D$ is preserved.
From a cryptographic perspective, we do not differentiate the confidentiality of ciphered data and plain text data, while the data is in transit in the network.
Based on our definition, the confidentiality of data is violated if an unauthorized user has been able to read it, and the fact that the data has been ciphered by the sender does not invalidate the violation from our point of view.
From this definition, we can describe several properties directly affecting the physical infrastructure or virtual networks.


% Give names instead of numbers 1, 2?
% node versus network element.  are they the same?

\textbf{Property 1a:} A physical node respects the confidentiality of the data it is carrying if and only if the user accessing the node and the data is authorized to do so.
%\RW{is the table helpful?  the formula seems clearer to me. rw}\FC{Cleaned}
\newline \textbf{Formal definition:} Let $N$ be a network element, $U$ a user (malicious or not), $D$ the data carried by the network, and $t$ the time.


\begin{myformula}
$node\_confidentiality(N) \Leftrightarrow \forall D,U,t~
confidentiality(D) \wedge is\_carrying(N,D,t) \wedge (accessing(U,N,t) \Rightarrow is\_node\_authorized(U,N,t))$
\end{myformula}

\begin{itemize}
\item $accessing(U,N,t)$ is true if user $U$ accesses network element $N$ at time $t$.
\item $is\_node\_authorized(U,N,t)$ is true if the user $U$ has the right to access network element $N$ at time $t$.
\item $is\_carrying(N,D,t)$ is true if node $N$ is carrying data $D$ at time $t$.
\end{itemize}

\textbf{Property 1b:} A virtual node respects the confidentiality of the data it is carrying if and only if the embedding nodes have their confidentiality respected as well.
%\RW{is the table helpful?  the formula seems clearer to me. rw}\FC{Cleaned}
\newline \textbf{Formal definition:} Let $n$ be a virtual node and $E_n$ the physical resources embedding $n$.


\begin{myformula}
$vnode\_confidentiality(n) \Leftrightarrow \bigwedge\limits_{N\in E_n} node\_confidentiality(N) $
\end{myformula}

\textbf{Property 1c:} A virtual link respects the confidentiality of the data it is carrying if and only if the embedding nodes respect the confidentiality of the data as well.
We remind the reader that the security of physical links is assumed here.
%\RW{is the table helpful?  the formula seems clearer to me. rw}\FC{Cleaned}
\newline \textbf{Formal definition:} Let $l$ be a virtual link and $E_l$ the physical resources embedding $l$.


\begin{myformula}
$vlink\_confidentiality(l) \Leftrightarrow \bigwedge\limits_{N\in E_l} node\_confidentiality(N) $
\end{myformula}

\textbf{Property 2:} A physical path preserves confidentiality from end to end if all of its network elements preserve their own confidentiality.
\newline \textbf{Formal definition:} Let $P$ be the physical path taken by the traffic, $L_P$ be the set of physical nodes composing $P$.

\begin{myformula}
$path\_confidentiality(P) \Leftrightarrow \bigwedge\limits_{N \in L_P}node\_confidentiality(N)$
\end{myformula}

% \textbf{Property 2b:} A virtual path preserves confidentiality from end to end if all of its virtual nodes and links preserve their own confidentiality.
% \newline \textbf{Formal definition:} Let $P$ be the path taken by the traffic, $L_N$ be the set of virtual nodes composing $P$, and $L_L$ the set of virtual links composing $P$.
% % and $L_{s}$ be the set of the links between each node of $P$.
% \begin{myformula}
% $vpath\_confidentiality(P) \Leftrightarrow \bigwedge\limits_{n \in L_N}vnode\_confidentiality(n) \wedge \bigwedge\limits_{l \in L_L}vlink\_confidentiality(l)$
% \end{myformula}

\textbf{Property 3:} A node's configuration remains confidential during a migration if the employed communication channel respects the confidentiality of the communication.
\newline \textbf{Formal definition:} Let $N$ be a node, $P$ a path, $conf$ be the configuration to be pushed, $C$ be an hypervisor.
\newline

\begin{myformula}
$config\_confidentiality(conf,N) \Leftrightarrow~\forall t,~\exists~C,P, 
is\_hypervisor\_of(C,N,t) \wedge is\_path(C,P,N,t) \wedge  conf\_of\_node(conf,N,t) \Rightarrow
~path\_confidentiality(P) \wedge  node\_confidentiality(N)
$
\end{myformula}


\begin{itemize}
\item $is\_hypervisor\_of(C,N,t)$ is true if hypervisor $C$ is affected to node $N$ at time $t$. 
\item $is\_path(C,P,N,t)$ is true if path $P$ is the path between $C$ and $N$ at time $t$.
\item $conf\_of\_node(conf,N,t)$ is true if node $N$ is running configuration $conf$ at time $t$.
\end{itemize}

\subsection{Integrity}
\label{sec:prop-int}
Preserving the integrity of information means protecting the accuracy and completeness of the information and the methods that are used to process and manage it~\cite{ISO/IEC270012013}.
We postulate that the meaning of accuracy and completeness in our context is equivalent to considering that information and processes have only been altered by an authorized user or not altered at all.
We emphasize on the fact that the integrity property is evaluated over a period of time instead of being instantaneous.
%\RW{unclear--why "therefore";  what are the temporal values of a piece of information? is that the result of eval at a given time?}\FC{Yes, the integrity of information means that it remains constant through time, unless processed, obviously}

We use the following notations for the integrity property:

\begin{itemize}
    \item $t_1$ and $t_2$ are dates
    \item $D_{t_1}$ (resp. $D_{t_2}$) is the value of data $D$ at date $t_1$ (resp. $t_2$)
    \item $P_{t_1}$ (resp. $P_{t_2}$) is the value of process $P$ at date $t_1$ (resp. $t_2$)
\end{itemize}

We define two predicates, $d\_integrity(D,ti)$ and $p\_integrity(P,D,ti)$, which represent  the integrity of the data $D$ itself and the integrity of the process $P$ used to manipulate $D$, over a time interval $ti$. 

We detail two notions related to the integrity property:
\begin{itemize}
    \item A data has been \textbf{legitimately processed} during a time interval if and only if it has been processed by the expected network nodes by the expected flow rules.
    \item A process has been \textbf{legitimately modified} if and only if it has been modified by an authorized network hypervisor.
\end{itemize}

% Section~\ref{sec:mapping-model} proposes solutions to verify these concepts.

Therefore, if a data has not been legitimately processed , its integrity holds if it has not been modified in any other way (\eg man-in-the-middle attack). 

\begin{myformula}
$d\_integrity(D,ti) \Leftrightarrow \forall t_1,t_2\in ti ~ \overline{processed(D, ti)} \Rightarrow 
~D_{t_1}=D_{t_2}$ 
\end{myformula}
%\RW{I changed the formatting for clarity}

\begin{itemize}
\item $processed(D, ti)$ holds true if $D$ has been legitimately processed during time interval $ti$.
% \item $deval(D,t)$ returns the value of $D$ at time $t$
%\RW{Does this mean that it is not defined if D has been legitimately processed?  Or it is true if D has been processed? or it does not hold if D has been processed?}\FC{It is still true if D has been legally processed}

\end{itemize}

Similarly, if a process has not been legitimately modified, its integrity holds if it has not been modified in any other way (\eg unauthorized flow rule modification). 

\begin{myformula}
$p\_integrity(P, ti) \Leftrightarrow \forall t_1,t_2 \in ti~ \overline{modified(P, ti)}\Rightarrow 
~P_{t_1}=P_{t_2}$
\end{myformula}

\begin{itemize}
\item $modified(P,ti)$ holds true if process $P$ has been modified legitimately during time interval $ti$.
% \item $peval(P,t)$ returns the process $P$ at time $t$
\end{itemize}
%\RW{you use P for paths and processes. and p for a function. what is the output of data? is it the value (eval?)}\FC{I don't understand}\GB{beware: you've been using $P$ for two distinct notions. I think that what RW means it that $p$ evaluates the realization of process $P$ over data $D$. In that sense you could, replace $p$ by the keyword $eval$}


\textbf{Property 1a:} The integrity of a network element is preserved if the data it carries and the process used to process it have their integrity preserved. 
\newline
\textbf{Formal definition:} Let $N$ be a network element, $D$ some data and $P$ a function modeling the process used by a network node on some data.

\begin{myformula}
$node\_integrity(N) \Leftrightarrow \forall D,ti,P~p\_integrity(P,ti) \wedge d\_integrity(D,ti) \wedge process\_of(P,N,ti) \wedge is\_carrying(N,D,ti) \Rightarrow is\_node\_authorized(U,N,ti)$
\end{myformula}

\begin{itemize}
\item $process\_of(P,N,ti)$ assigns process $P$ to node $N$ during time interval ti.
\item $is\_carrying(N,D,ti)$ is true if node $N$ is carrying data $D$ during time interval $ti$.
\item $is\_node\_authorized(U,N,ti)$ is true if the user $U$ has the right to access network element $N$ during time interval $ti$.
\end{itemize}

\textbf{Property 1b:} The integrity of a virtual node is preserved if and only if the nodes embedding it preserve their integrity as well.
\newline
\textbf{Formal definition:} Let $n$ be a virtual node and $E_n$ the set of nodes embedding $n$.

\begin{myformula}
$vnode\_integrity(n) \Leftrightarrow \bigwedge\limits_{N \in E_n} node\_integrity(N)$
\end{myformula}

\textbf{Property 1c:} The integrity of a virtual link is preserved if and only if the nodes embedding it preserve their integrity as well.
\newline
\textbf{Formal definition:} Let $l$ be a virtual link and $E_l$ the set of nodes embedding $l$.

\begin{myformula}
$vlink\_integrity(l) \Leftrightarrow \bigwedge\limits_{N \in E_l} node\_integrity(N)$
\end{myformula}


\textbf{Property 2:} A path preserves integrity from end to end if all of its network elements preserve their own integrity.
\newline \textbf{Formal definition:} Let $P$ be a path in the network and $L_P$ be the set of nodes composing $P$.
\newline

\begin{myformula}
$path\_integrity(P) \Leftrightarrow \bigwedge\limits_{N \in L_P}integrity(N) $
\end{myformula}

% \textbf{Property 2b:} A virtual path preserves integrity from end to end if all its virtual nodes and links preserve their own integrity.
% \newline \textbf{Formal definition:} Let $P$ be a virtual path, $L^P_N$ be the set of virtual nodes composing $P$ and $L_L$ be the set of virtual links composing $P$ .
% \newline

% \begin{myformula}
% $vpath\_integrity(P) \Leftrightarrow \bigwedge\limits_{n \in L^P_N}vnode\_integrity(n) \wedge \bigwedge\limits_{l \in L_L}vlink\_integrity(l) $
% \end{myformula}

\textbf{Property 3:} A node's configuration maintains its integrity during a migration if it is not altered during the installation from the hypervisor to the node.
\newline \textbf{Formal definition:} Let $N$ be a node, $conf$ be the configuration data to be pushed on node $N$, $C$ be a network hypervisor.
\newline
%\RW{ i proposed a replacement but i still have questions about it:  $confidentiality(conf,N) \Leftrightarrow \exists C,P~ confidentiality(C) \wedge confidentiality(P) \wedge confidentiality(N) \wedge is\_controller\_of(C,N)$}
%\RW{needs quantifiers for C U P N.  also, the formula does not constrain these to have anything to do with conf it seems that to define a confidentiality-preserving migration requires mentioning time.  the network is confidential after the migration if it was confidential before.  }

\begin{myformula}
$config\_integrity(conf,N) \Leftrightarrow~\forall ti,\exists~C,P 
~is\_hypervisor\_of(C,N,ti) \wedge is\_path(C,P,N,ti) \wedge  conf\_of\_node(conf,N,ti) \Rightarrow
~d\_integrity(conf,ti) \wedge path\_integrity(P)$
\end{myformula}

\begin{itemize}
\item $is\_hypervisor\_of(C,N,ti)$ is true if hypervisor $C$ is affected to node $N$ during time interval $ti$.  
\item $is\_path(C,P,N,ti)$ is true if path $P$ is the path between $C$ and $N$ during time interval $ti$.
\item $conf\_of\_node(conf, N,ti)$ is true if node $N$ is running configuration $conf$ during time interval $ti$.
\end{itemize}

\subsection{Availability}
\label{sec:prop-avail}
An asset is available if it is accessible and usable when
needed by an authorized entity~\cite{ISO/IEC270012013}.

We can define the predicate $availability(A)$ representing the availability of asset $A$.
\newline
Let $A$ be an asset, $U$ a user and $ti$ a time interval.
\newline
%\RW{why would availability be discussed without specifying a time?  it seems unlikely that an asset is always useable.}
%\FC{In the same way we consider confidentiality and integrity, as long availability is preserved we don't need time, but we need to pinpoint the time intervals related to the violation if needed.}

\begin{myformula}
$availability(A) \Leftrightarrow \forall U,ti~is\_accessible(U,A,ti) \wedge is\_usable(U,A,ti) \Rightarrow is\_asset\_authorized(U,A,ti)$
\end{myformula}

% \RW{why would an asset not be available, if it is useable and the user is authorized?  is it because an authorized path does not exist?}
% \FC{I've added authorization}
\begin{itemize}
\item $is\_accessible(U,A,ti)$ is true if user $U$ is able to contact asset $A$ during time interval $ti$.
\item $is\_usable(U,A,ti)$ is true if user $U$ is able to use asset $A$ during time interval $ti$.
\item $is\_asset\_authorized(U,A,ti)$ is true if user $U$ is allowed to access asset $A$  during time interval $ti$.
\end{itemize}

\textbf{Property 1a:} A network node $N$ is available if it can be reached and it can process incoming packets.
\newline
\textbf{Formal definition:} Let $N$ be a network node and $ti$ be a time interval.

\begin{myformula}
$node\_availability(N) \Leftrightarrow \forall ti,U~ is\_accessible(U,N,ti) \wedge can\_process(N,ti) \Rightarrow is\_node\_authorized(U,N,ti)$
\end{myformula}

\begin{itemize}
\item $is\_accessible(U,N,ti)$ is true if user $U$ can contact node $N$ during time interval $ti$.
\item $can\_process(N,ti)$ is true if $N$ processes packets normally during time interval $ti$ (\eg no congestion etc.).
\item $is\_node\_authorized(U,N,ti)$ is true if user $U$ is authorized to access node $N$ during time interval $ti$.
\end{itemize} 

\textbf{Property 1b:} A virtual node $n$ is available if the physical nodes embedding it are available.
\newline
\textbf{Formal definition:} Let $n$ be a virtual node and $E_n$ the nodes embedding it.

\begin{myformula}
$vnode\_availability(n) \Leftrightarrow \bigwedge\limits_{N \in E_n} node\_availability(N)$
\end{myformula}

\textbf{Property 1c:} A virtual link $l$ is available if the physical nodes embedding it are available.
\newline
\textbf{Formal definition:} Let $l$ be a virtual link and $E_l$ the nodes embedding it.

\begin{myformula}
$vlink\_availability(l) \Leftrightarrow \bigwedge\limits_{N \in E_l} node\_availability(N)$
\end{myformula}

\textbf{Property 2:} A path is available from end to end if all of its nodes are available.
\newline \textbf{Formal definition:} Let $P$ be the path taken by the traffic, $L_P$ be the set of nodes composing $P$. 

\begin{myformula}
$path\_availability(P)\Leftrightarrow\bigwedge\limits_{N \in L_P}node\_availability(N)$
\end{myformula}


% \textbf{Property 2b:} A virtual path is available from end to end if all of its virtual nodes and virtual links are available as well.
% \newline \textbf{Formal definition:} Let $P$ be a virtual path, $L^P_N$ be the set of virtual nodes composing $P$ and $L_L$ the set of the virtual links composing $P$.

% \begin{myformula}
% $vpath\_availability(P)\Leftrightarrow \bigwedge\limits_{n \in L^P_N}vnode\_availability(n) \wedge \bigwedge\limits_{l \in L_L}vlink\_availability(l)$
% \end{myformula}

% \subsection{Authenticity}
% \label{sec:prop-auth}
% Authenticity is a property or characteristic of an entity.
% An entity is authentic if it is what it claims to be~\cite{ISO/IEC270012013}.

% Let $E$ be an entity, $R$ a role for $E$ and $V$ the verification authority.
% The authenticity of an entity $E$ can be defined as follow:

% \begin{myformula}
% $ authenticity(E) \Leftrightarrow \forall t, claim\_role(E,t) \wedge has\_role(E,V,t)$
% \end{myformula}
% \begin{itemize}
% \item claim\_role(E,t) returns the role R of E at time t
% \item has\_role(E,V,t) the role R of E is returned by V at time t
% \end{itemize}
% % \RW{why is authenticity defined without a time argument but claim\-role does have a time argument?}
% % \FC{Similarly to confidentiality, authenticity is a binary property that cannot be restored}\GB{what do you mean by ``cannot be restored''?}
% % \FC{For authenticity, once the user lied about who he is and we found out, you cannot say "I will still allow you to user our network."}

% \textbf{Property 1:} The authenticity of a migration request is verified if the topology migrated belongs to the requesting user, and the user is authentic.
% \newline
% \textbf{Formal definition: } Let $U$ be a user, $V$ a virtual topology, $r$ the migration request and $N$ the network hypervisor.

% \begin{myformula}
% $migration\_authenticity(r,U) \Leftrightarrow \exists U,~authenticity(U) \wedge belongs\_to(r_V,U)$
% \end{myformula}
% %\RW{there is no quantifier for V and it is not a parameter.}
% %\FC{V is a parameter inside request r²}
% \begin{itemize}
% \item $authenticity(U)$ is true if $U$ is a tenant known by $N$
% \item $r_V$ is the topology concerned by request $r$
% \item $belongs\_to(r_V,U)$ is true if the topology $V$ belongs to user $U$
% \end{itemize}
% %\RW{does it matter what r is in this definition?}
% %\FC{r is just the request, it's not under the scope for examination}


\subsection{Co-Residency}
\label{sec:prop-cores}
The co-residency between two virtual networks represents how much common physical resources they share.
Precisely, two virtual networks are co-residing together if they are embedded, partially or totally, in the same physical resources during the same time interval.
We define co-residency with the following predicate:
Let $V_1$ and $V_2$ be two virtual topologies belonging to different users, $P_1$ and $P_2$ two sets of physical resources and let $ti$ be a time interval.
\begin{myformula}
$coresidency(V_1,V_2,ti) \Leftrightarrow \exists P_1,P_2$~s.t.~$P_1 \cap P_2~\neq~\emptyset, $\\ $is\_embedding(P_1,V_1,t_i) \wedge is\_embedding(P_2,V_2,t_i)$
\end{myformula}

\begin{itemize}
    \item $is\_embedding(P,V,t_i)$ is true if the physical set of resources $P$ embeds the virtual network $V$ during time interval $t_i$.
\end{itemize}

Co-residency is a property slightly different from the others because it is not directly related to security.
However, as shown in Chapter~\ref{sec:sota}, co-residency is an important metric for the success of an attack.
Indeed, side channel attacks on Virtual Machines often rely on the co-residency of both the victim's and the attacker's VMs to succeed.



\subsection{Mapping formal predicates with technical events}
\label{sec:mapping-model}
In the previous sections we have described the different properties we want to observe inside the SDN infrastructure.
We propose a mapping between the formal predicates and the corresponding SDN events or other technical situations.


\subsubsection{Confidentiality}\textbf{\\}
The confidentiality section introduces the following predicates:
\begin{itemize}
\item reads(U,D,t)
\newline
This predicate is true if a monitoring tool intercepts traffic, or if the traffic is forwarded to a virtual machine, owned by either a legitimate or malicious user. Inspecting flow rules in nodes determines that. 

\item is\_data\_authorized(U,D,t)
\newline
This predicate is true if user U is the intended recipient of data D (\eg destination IP).
D can be inspected with monitoring tools to verify the destination IP field.
\item is\_node\_authorized(U,N,t)
\newline
This predicate is true if user U is authorized to access the physical node N.
This is a typical use of Access Control.
\item accessing(U,N,t)
\newline
This predicate is true if U logs in N using a CLI or use remote access to interact with N.
The logs of N can be exploited to verify the accesses made.
It is important to note that installing flow rules on a node is not considered an access in the meaning of this predicate.

\item is\_carrying(N,D,t)\\
This predicate is verified by checking if the flow rules composing the path taken by data D go through node N.

\item is\_hypervisor\_of(C,N,t)\\
This predicate is verified by the hypervisor by checking if N has established a connection.

\item is\_path(C,P,N,t)\\
This predicate  is verified by the hypervisor by checking the network topology.

\item conf\_of\_node(conf,N,t)\\
This predicate  is verified by checking the flow rules of node N.

\end{itemize}

\subsubsection{Integrity}\textbf{\\}
The integrity section introduces the following predicates:
\begin{itemize}

\item processed(D,ti)
\newline
If D was not in transit during ti then the predicate holds true.
Otherwise, the flow rules used to process D during ti are verified by the hypervisor by querying network nodes, and D can be analyzed to see if it matches the expected results. (\ie no extra illegitimate manipulations have been performed).

\item modified(P,ti)
\newline
The evolution of P implies legitimate modification of the corresponding flow rules, which can be monitored by the hypervisor. The hypervisor may store rules and legitimate requests sent to modify them. Any unexpected discrepancy will yield False for this predicate.

\item is\_carrying(N,D,ti)\\
This predicate is verified by checking if the flow rules composing the path taken by data D go through node N.


% \item deval(D,t)
\item $D_{t}$
\newline
This is done by reading the value of D using a network probe.

\item $P_t$
\newline
The process P can be monitored by querying the related flow rules

\item proccess\_of(P,N,ti)
\newline
The flow rules of node N can be queried to check if they matches process P.

\item is\_node\_authorized(U,N,ti)\\
% This predicate is true if user U is authorized to access the physical node N.
This is a typical use of Access Control.

\item is\_hypervisor\_of(C,N,t)\\
This predicate is verified using the list of nodes connected to the hypervisor.

\item is\_path(C,P,N,ti)\\
This predicate  is verified by the hypervisor and its view of the physical topology.

\item conf\_of\_node(conf,N,ti)\\
This predicate  is verified by checking the flow rules of node N,

\end{itemize}

\subsubsection{Availability}\textbf{\\}
The availability section introduces the following predicates:
\begin{itemize}
\item is\_accessible(U,A,ti)
\\ This predicate is evaluated by the SDN controller regularly sending a keep-alive request to the asset.

\item is\_accessible(U,N,ti)
\\ This predicate is evaluated by the SDN controller regularly sending a keep-alive request to the node.


\item is\_usable(U,A,ti)
\\This predicate is evaluated by monitoring the performance of A with monitoring tools .


\item can\_process(N,ti)
\\ OpenFlow switches provide performance statistics, and switches have primitives to show the amount of CPU power used.

\item is\_node\_authorized(U,N,ti)\\
This predicate is true if user U is authorized to access the physical node N.
This is a typical use of Access Control.

\item is\_asset\_authorized(U,A,ti)\\
This predicate is verified using Access Control as well.
\end{itemize}

% \subsubsection{Authenticity}
% The authenticity section introduces the following predicates:
% \begin{itemize}

% \item claim\_role(E)
% \item has\_role(E,V)
% \\
% Those two predicates are verified by the network hypervisor upon receiving a migration requests.
% The DPAC will provide an interface to identify tenants to let them interact with the hypervisor.
% \item belongs\_to($r_V$,U)
% \\
% The network hypervisor keeps a mapping between virtual topologies and their owner, hence it is trivial to verify this predicate.
% \end{itemize}
 
 
\subsubsection{Co-Residency}\textbf{\\}
The Co-Residency section introduces the following predicate:

\begin{itemize}
    \item is\_embedding(P,V,ti) is verified by the hypervisor by checking the embedding he deployed. Alternatively, checking the corresponding flow rules of the nodes in P verifies this predicate.
\end{itemize}