We describe here the predicates used to represent the physical network and how virtual networks are embedded on it.

For an eased reading of the rest of this section, we define some variable names:

\begin{itemize}
    \item \textbf{hname} is the unique identifier of a network hypervisor
    \item \textbf{vnet} is the unique identifier of a virtual network
    \item \textbf{pnode\_name} is the unique identifier of a physical node
    \item \textbf{plink\_name} is the unique identifier of a physical link
    \item \textbf{ppath\_name} is the unique identifier of a physical path
    \item \textbf{vnode\_name} is the unique identifier of a virtual node
    \item \textbf{vlink\_name} is the unique identifier of a virtual link
    % \item \textbf{vpath\_name} is the unique identifier of a virtual path
    \item \textbf{ti} is the unique identifier of a time interval.
\end{itemize}

Each variable is distinct from each other by adding an integer at the end.

\subsubsection{Physical network modeling}


The physical network is composed of nodes and links. We define two temporal predicates:

physical\_node(\textbf{pnode\_name1},\textbf{ti1})

This predicate declares the existence of a physical node named \textbf{pnode\_name1} and the node exists during time interval \textbf{ti1}.

physical\_link(\textbf{plink\_name1},\textbf{hname1}$\vert$\textbf{pnode\_name1},\textbf{pnode\_name2},\textbf{ti1})

This predicate declares the existence of a physical link named \textbf{plink\_name1}, linking physical node \textbf{pnode\_name1} (or hypervisor \textbf{hname1}) with \textbf{pnode\_name2}, and the link exists during time interval \textbf{ti1}.

physical\_path(\textbf{ppath\_name1},\textbf{pnode\_name1}$\vert$\textbf{hname1},\textbf{pnode\_name2},\textbf{ti1})

This predicate declares the existence of a physical path named \textbf{ppath\_name1} connecting \textbf{pnode\_name1} (or hypervisor \textbf{hname1}) with \textbf{pnode\_name2}, and the path exists during time interval \textbf{ti1}. 

Similarly to a physical link connecting two nodes, a physical path represent the set of nodes and links establishing the connection between two end points that may be separated by several physical nodes and links.
The case of a physical path directly connecting two nodes (or a node and an hypervisor) is allowed.

A physical path is given a unique identifier and is associated with a set of physical links. 
This choice of modeling allows to distinguish the different paths possible to connect two nodes. 
We represent this association with a predicate:

part\_of\_ppath(\textbf{ppath\_name1},\textbf{plink\_name1},\textbf{ti1})

This predicates associates physical link \textbf{plink\_name1} to physical path \textbf{ppath\_name1} during time \textbf{ti1}.

We illustrate physical\_path(\textbf{ppath\_name1},\textbf{pnode\_name1},\textbf{pnode\_name3},\textbf{ti1}) with a toy example:

physical\_node(\textbf{pnode\_name1},\textbf{ti1})\\
physical\_node(\textbf{pnode\_name2},\textbf{ti1})\\
physical\_node(\textbf{pnode\_name3},\textbf{ti1})\\
physical\_link(\textbf{plink\_name1},\textbf{pnode\_name1},\textbf{pnode\_name2},\textbf{ti1})\\
physical\_link(\textbf{plink\_name2},\textbf{pnode\_name2},\textbf{pnode\_name3},\textbf{ti1})\\
physical\_path(\textbf{ppath\_name1},\textbf{pnode\_name1},\textbf{pnode\_name3},\textbf{ti1})\\
part\_of\_ppath(\textbf{ppath\_name1},\textbf{plink\_name1},\textbf{ti1})\\
part\_of\_ppath(\textbf{ppath\_name1},\textbf{plink\_name2},\textbf{ti1})

Therefore physical\_path(\textbf{ppath\_name1},\textbf{pnode\_name1},\textbf{pnode\_name3},\textbf{ti1}) holds \textbf{true} because the part\_of\_ppath predicates build a valid path.

% \textbf{Precisions on the use of time intervals}

% The time interval used in these predicates represents the duration during which the predicates holds true.

% For instance, if node1 has a failure at $t_1$ and only comes back available again at $t_2$ until $t_3$. We obtain the following predicates:


% physical\_node(node1,[0,t\_1])
% physical\_node(node1,[t\_2,t\_3])


% $t_3$ could be seen as $+\infty$ if no other incident occurs on node1.


\subsubsection{Network Hypervisor modeling}
The network hypervisor is connected to the different physical nodes, administrates them, installs flow rules \etc.

hypervisor(\textbf{hname1},\textbf{ti1})

This predicate declares the existence of the hypervisor \textbf{hname1} during time interval \textbf{ti1}.

hypervisor\_of(\textbf{hname1},\textbf{pnode\_name1},\textbf{ppath\_name1},\textbf{ti1})

This predicate affects hypervisor \textbf{hname1} to the physical node \textbf{pnode\_name1}, and sets \textbf{ppath\_name1} as the physical path between them during time interval \textbf{ti1}.

The network hypervisor establishes a communication with the nodes in the infrastructure using physical paths.
These paths are illustrated by the "Hypervisor - Switch links`` as depicted in Figure~\ref{fig:VNE-example-model}.
In this figure the hypervisor is directly connected to all the nodes of the infrastructure.



\subsubsection{Virtual network modeling}
We extend the previous predicates with information related to the tenant of the virtual network and how the VN is embedded on top of a set of physical resources.

virtual\_network(\textbf{vnet\_name1},\textbf{ti1})

This predicate declares the existence of a virtual network \textbf{vnet\_name1} during time interval \textbf{ti1}.

virtual\_node(\textbf{vnode\_name1},\textbf{vnet\_name1},\textbf{ti})

This predicate declares the existence of a virtual node \textbf{vnode\_name1} belonging to virtual network \textbf{vnet\_name1} during time interval \textbf{ti1}.

virtual\_link(\textbf{vlink\_name1},\textbf{vnode\_name1},\textbf{vnode\_name2},\textbf{vnet\_name1},\textbf{ti1})

This predicate declares the existence of a virtual link \textbf{vnode\_link1} between \textbf{vnode\_name1} and \textbf{vnode\_name2} belonging to virtual network \textbf{vnet\_name1} during time interval \textbf{ti1}.

% virtual\_path(\textbf{vpath\_name1},\textbf{vnode\_name1},\textbf{vnode\_name2},\textbf{ti1})

\subsubsection{Embedding modeling}
Each virtual element, node or link, is embedded on a set of physical resources.
Therefore, there can be one or more predicates used to describe the embedding of a virtual element.


node\_embedding(\textbf{vnode\_name1},\textbf{pnode\_name1},\textbf{ti1})

This predicate affects physical node \textbf{pnode\_name1} to the embedding of virtual node \textbf{vnode\_name1} during time interval \textbf{ti1}.

node\_embedding(\textbf{vnode\_name1},\textbf{plink\_name1},\textbf{ti1})

This predicate affects physical link \textbf{plink\_name1} to the embedding of virtual node \mbox{\textbf{vnode\_name1}} during time interval \textbf{ti1}.

link\_embedding(\textbf{vlink\_name1},\textbf{pnode\_link1},\textbf{ti1})

This predicate affects physical link \textbf{plink\_name1} to the embedding of virtual link \textbf{vnode\_link1} during time interval \textbf{ti1}.

We outline that a virtual link is a succession of physical links, therefore we do not model its embedding using a single physical node.


\subsubsection{Example for modeling physical and virtual networks}

We consider the example depicted in Fig.~\ref{fig:VNE-example-model}.
There is no failure in the system nor any attack, therefore, we use the time interval \textbf{ti1} to represent the duration of the predicates.

\begin{figure}[ht]
\centering

\tikzset{every picture/.style={line width=0.75pt}} %set default line width to 0.75pt        

\begin{tikzpicture}[x=0.75pt,y=0.75pt,yscale=-1,xscale=1]
%uncomment if require: \path (0,688.6666717529297); %set diagram left start at 0, and has height of 688.6666717529297

%Rounded Rect [id:dp6173726728290502] 
\draw  [fill={rgb, 255:red, 242; green, 175; blue, 175 }  ,fill opacity=1 ] (274,105.23) .. controls (274,94.98) and (282.31,86.67) .. (292.57,86.67) -- (506.43,86.67) .. controls (516.69,86.67) and (525,94.98) .. (525,105.23) -- (525,160.93) .. controls (525,171.19) and (516.69,179.5) .. (506.43,179.5) -- (292.57,179.5) .. controls (282.31,179.5) and (274,171.19) .. (274,160.93) -- cycle ;
%Straight Lines [id:da6819841257680012] 
\draw    (318,148.5) -- (479,143.67) ;


%Image [id:dp2340333324734869] 
\draw (309,147.5) node  {\includegraphics[width=52.5pt,height=52.5pt]{figures/router-158644_1280.pdf}};
%Image [id:dp04055959421826061] 
\draw (394.5,147.5) node  {\includegraphics[width=52.5pt,height=52.5pt]{figures/router-158644_1280.pdf}};
%Image [id:dp9259427266230486] 
\draw (484,147.5) node  {\includegraphics[width=52.5pt,height=52.5pt]{figures/router-158644_1280.pdf}};

%Rounded Rect [id:dp4478547267123999] 
\draw  [fill={rgb, 255:red, 255; green, 248; blue, 177 }  ,fill opacity=1 ] (26,125.27) .. controls (26,105.05) and (42.39,88.67) .. (62.6,88.67) -- (178.4,88.67) .. controls (198.61,88.67) and (215,105.05) .. (215,125.27) -- (215,235.07) .. controls (215,255.28) and (198.61,271.67) .. (178.4,271.67) -- (62.6,271.67) .. controls (42.39,271.67) and (26,255.28) .. (26,235.07) -- cycle ;
%Straight Lines [id:da04812807065340363] 
\draw    (56,180.67) -- (166,134.67) ;


%Image [id:dp341968705695737] 
\draw (61,184.5) node  {\includegraphics[width=52.5pt,height=52.5pt]{figures/router-29825_1280.pdf}};
%Image [id:dp7725023918834623] 
\draw (165,130.5) node  {\includegraphics[width=52.5pt,height=52.5pt]{figures/router-29825_1280.pdf}};

%Straight Lines [id:da036026819015278266] 
\draw    (81,200.17) -- (138,225.67) ;


%Image [id:dp29149867233304716] 
\draw (136,238.5) node  {\includegraphics[width=52.5pt,height=52.5pt]{figures/router-29825_1280.pdf}};

%Rounded Rect [id:dp8669891034149613] 
\draw  [fill={rgb, 255:red, 184; green, 233; blue, 134 }  ,fill opacity=1 ] (8,398.47) .. controls (8,383.11) and (20.45,370.67) .. (35.8,370.67) -- (583.2,370.67) .. controls (598.55,370.67) and (611,383.11) .. (611,398.47) -- (611,481.87) .. controls (611,497.22) and (598.55,509.67) .. (583.2,509.67) -- (35.8,509.67) .. controls (20.45,509.67) and (8,497.22) .. (8,481.87) -- cycle ;
%Straight Lines [id:da7257050611297167] 
\draw    (79,428.67) -- (214,401.67) ;


%Straight Lines [id:da6984840894149047] 
\draw    (80,443.67) -- (171,466.67) ;


%Straight Lines [id:da6802535311385733] 
\draw    (174,473.67) -- (368,426.67) ;


%Straight Lines [id:da14214857377238777] 
\draw    (214,401.67) -- (358,410.67) ;


%Straight Lines [id:da6556428261701027] 
\draw    (384,420.67) -- (502,407.67) ;


%Straight Lines [id:da16066583484396146] 
\draw    (375,440.67) -- (477,476.67) ;


%Image [id:dp1703433460369157] 
\draw (70,443.5) node  {\includegraphics[width=52.5pt,height=52.5pt]{figures/router-30140_1280.pdf}};
%Image [id:dp7119657242561229] 
\draw (160,474.5) node  {\includegraphics[width=52.5pt,height=52.5pt]{figures/router-30140_1280.pdf}};
%Image [id:dp01793726972424281] 
\draw (200,411.5) node  {\includegraphics[width=52.5pt,height=52.5pt]{figures/router-30140_1280.pdf}};
%Image [id:dp9361502418924952] 
\draw (386,425.5) node  {\includegraphics[width=52.5pt,height=52.5pt]{figures/router-30140_1280.pdf}};
%Image [id:dp1834582453825948] 
\draw (515,414.5) node  {\includegraphics[width=52.5pt,height=52.5pt]{figures/router-30140_1280.pdf}};
%Image [id:dp34021892319279323] 
\draw (515,487.5) node  {\includegraphics[width=52.5pt,height=52.5pt]{figures/router-30140_1280.pdf}};
%Straight Lines [id:da9221433559107663] 
\draw [line width=1.5]  [dash pattern={on 1.69pt off 2.76pt}]  (63.05,209) -- (62,408.97) ;


%Straight Lines [id:da18979728678812846] 
\draw [line width=1.5]  [dash pattern={on 1.69pt off 2.76pt}]  (139.05,261) -- (140,442.97) ;


%Straight Lines [id:da3949095016333407] 
\draw [line width=1.5]  [dash pattern={on 1.69pt off 2.76pt}]  (178.05,153) -- (186.2,381.27) ;


%Straight Lines [id:da9197957685386293] 
\draw [line width=1.5]  [dash pattern={on 1.69pt off 2.76pt}]  (314.05,169.17) -- (84,411) ;


%Straight Lines [id:da5299009760945013] 
\draw [line width=1.5]  [dash pattern={on 1.69pt off 2.76pt}]  (492.05,168) -- (495.6,384.07) ;


%Straight Lines [id:da535589916512549] 
\draw [line width=1.5]  [dash pattern={on 1.69pt off 2.76pt}]  (390.05,170) -- (390.4,391.27) ;


%Rounded Rect [id:dp33971726070681063] 
\draw  [fill={rgb, 255:red, 217; green, 154; blue, 232 }  ,fill opacity=1 ] (49,306.78) .. controls (49,301.88) and (52.98,297.9) .. (57.89,297.9) -- (543.11,297.9) .. controls (548.02,297.9) and (552,301.88) .. (552,306.78) -- (552,333.45) .. controls (552,338.35) and (548.02,342.33) .. (543.11,342.33) -- (57.89,342.33) .. controls (52.98,342.33) and (49,338.35) .. (49,333.45) -- cycle ;

%Straight Lines [id:da3032548642460361] 
\draw [color={rgb, 255:red, 74; green, 144; blue, 226 }  ,draw opacity=1 ][line width=1.5]    (222.05,340.17) -- (208.13,376.37) ;
\draw [shift={(207.05,379.17)}, rotate = 291.04] [color={rgb, 255:red, 74; green, 144; blue, 226 }  ,draw opacity=1 ][line width=1.5]    (14.21,-4.28) .. controls (9.04,-1.82) and (4.3,-0.39) .. (0,0) .. controls (4.3,0.39) and (9.04,1.82) .. (14.21,4.28)   ;

%Straight Lines [id:da5649833646600878] 
\draw [color={rgb, 255:red, 74; green, 144; blue, 226 }  ,draw opacity=1 ][line width=1.5]    (330.05,344.17) -- (369.08,388.91) ;
\draw [shift={(371.05,391.17)}, rotate = 228.9] [color={rgb, 255:red, 74; green, 144; blue, 226 }  ,draw opacity=1 ][line width=1.5]    (14.21,-4.28) .. controls (9.04,-1.82) and (4.3,-0.39) .. (0,0) .. controls (4.3,0.39) and (9.04,1.82) .. (14.21,4.28)   ;

%Straight Lines [id:da5842761266197537] 
\draw [color={rgb, 255:red, 74; green, 144; blue, 226 }  ,draw opacity=1 ][line width=1.5]    (479.05,343.17) -- (485.53,380.21) ;
\draw [shift={(486.05,383.17)}, rotate = 260.07] [color={rgb, 255:red, 74; green, 144; blue, 226 }  ,draw opacity=1 ][line width=1.5]    (14.21,-4.28) .. controls (9.04,-1.82) and (4.3,-0.39) .. (0,0) .. controls (4.3,0.39) and (9.04,1.82) .. (14.21,4.28)   ;

%Straight Lines [id:da44476437360701715] 
\draw [color={rgb, 255:red, 74; green, 144; blue, 226 }  ,draw opacity=1 ][line width=1.5]    (93.05,345.17) -- (75.9,403.29) ;
\draw [shift={(75.05,406.17)}, rotate = 286.44] [color={rgb, 255:red, 74; green, 144; blue, 226 }  ,draw opacity=1 ][line width=1.5]    (14.21,-4.28) .. controls (9.04,-1.82) and (4.3,-0.39) .. (0,0) .. controls (4.3,0.39) and (9.04,1.82) .. (14.21,4.28)   ;

%Straight Lines [id:da5384124591664345] 
\draw [line width=1.5]  [dash pattern={on 1.69pt off 2.76pt}]  (445.5,16.33) -- (484.5,16.83) ;


%Straight Lines [id:da3606010400838229] 
\draw    (445,32.71) -- (485,33.21) ;


%Straight Lines [id:da015844815365207543] 
\draw [color={rgb, 255:red, 74; green, 144; blue, 226 }  ,draw opacity=1 ][line width=1.5]    (448.5,65.13) -- (478.5,64.97) ;
\draw [shift={(481.5,64.96)}, rotate = 539.71] [color={rgb, 255:red, 74; green, 144; blue, 226 }  ,draw opacity=1 ][line width=1.5]    (14.21,-4.28) .. controls (9.04,-1.82) and (4.3,-0.39) .. (0,0) .. controls (4.3,0.39) and (9.04,1.82) .. (14.21,4.28)   ;

%Straight Lines [id:da6059698523065559] 
\draw [color={rgb, 255:red, 74; green, 144; blue, 226 }  ,draw opacity=1 ][line width=1.5]    (161,343) -- (155.18,439.01) ;
\draw [shift={(155,442)}, rotate = 273.47] [color={rgb, 255:red, 74; green, 144; blue, 226 }  ,draw opacity=1 ][line width=1.5]    (14.21,-4.28) .. controls (9.04,-1.82) and (4.3,-0.39) .. (0,0) .. controls (4.3,0.39) and (9.04,1.82) .. (14.21,4.28)   ;

%Straight Lines [id:da7790467806562547] 
\draw [color={rgb, 255:red, 74; green, 144; blue, 226 }  ,draw opacity=1 ][line width=1.5]    (433,343) -- (487.71,458.29) ;
\draw [shift={(489,461)}, rotate = 244.61] [color={rgb, 255:red, 74; green, 144; blue, 226 }  ,draw opacity=1 ][line width=1.5]    (14.21,-4.28) .. controls (9.04,-1.82) and (4.3,-0.39) .. (0,0) .. controls (4.3,0.39) and (9.04,1.82) .. (14.21,4.28)   ;


% Text Node
\draw (85,498.5) node [scale=0.9] [align=left] {Physical Infrastructure};
% Text Node
\draw (59,137) node  [align=left] {V1};
% Text Node
\draw (119,117) node  [align=left] {V2};
% Text Node
\draw (144,188) node  [align=left] {V3};
% Text Node
\draw (353,111) node  [align=left] {V4};
% Text Node
\draw (437,113) node  [align=left] {V5};
% Text Node
\draw (512,108) node  [align=left] {V6};
% Text Node
\draw (54,475) node  [align=left] {P1};
% Text Node
\draw (205,486) node  [align=left] {P2};
% Text Node
\draw (244,390) node  [align=left] {P3};
% Text Node
\draw (342,397) node  [align=left] {P4};
% Text Node
\draw (561,394) node  [align=left] {P5};
% Text Node
\draw (565,473) node  [align=left] {P6};
% Text Node
\draw (531,16) node  [align=left] {Virtual Link};
% Text Node
\draw (538,37) node  [align=left] {Physical Link};
% Text Node
\draw (572,59) node  [align=left] {Hypervisor - Switch link};
% Text Node
\draw (300.5,320.11) node  [align=left] {Network Hypervisor : hname1};
% Text Node
\draw (69,104.5) node  [align=left] {{\small vSDN1}};
% Text Node
\draw (309,97.5) node [scale=0.9] [align=left] {{\small vSDN2}};


\end{tikzpicture}

\caption{Example of modeling the infrastructure}
\label{fig:VNE-example-model}
\end{figure}

The current infrastructure gives the following formal declaration:


virtual\_network(vSDN1,ti1)\\
virtual\_network(vSDN2,ti1)\\
virtual\_node(V1,vSDN1,ti1)\\
virtual\_node(V2,vSDN1,ti1)\\
virtual\_node(V3,vSDN1,ti1)\\
virtual\_node(V4,vSDN2,ti1)\\
virtual\_node(V5,vSDN2,ti1)\\
virtual\_node(V6,vSDN2,ti1)\\
virtual\_link(vlink1,V1,V2,vSDN1,ti1)\\
virtual\_link(vlink2,V1,V3,vSDN1,ti1)\\
virtual\_link(vlink3,V4,V5,vSDN2,ti1)\\
virtual\_link(vlink4,V5,V6,vSDN2,ti1)\\
physical\_node(P1,ti1)\\
physical\_node(P2,ti1)\\
physical\_node(P3,ti1)\\
physical\_node(P4,ti1)\\
physical\_node(P5,ti1)\\
physical\_node(P6,ti1)\\
physical\_link(plink1,P1,P2,ti1)\\
physical\_link(plink2,P1,P3,ti1)\\
physical\_link(plink3,P2,P4,ti1)\\
physical\_link(plink4,P3,P4,ti1)\\
physical\_link(plink5,P4,P5,ti1)\\
physical\_link(plink6,P4,P6,ti1)\\
physical\_path(ppath1,P1,hname1,ti1)\\
physical\_path(ppath2,P2,hname1,ti1)\\
physical\_path(ppath3,P3,hname1,ti1)\\
physical\_path(ppath4,P4,hname1,ti1)\\
physical\_path(ppath5,P5,hname1,ti1)\\
physical\_path(ppath6,P6,hname1,ti1)\\
hypervisor(hyp1,ti1)\\
hypervisor\_of(hyp1,P1,ppath1,ti1)\\
hypervisor\_of(hyp1,P2,ppath2,ti1)\\
hypervisor\_of(hyp1,P3,ppath3,ti1)\\
hypervisor\_of(hyp1,P4,ppath4,ti1)\\
hypervisor\_of(hyp1,P5,ppath5,ti1)\\
hypervisor\_of(hyp1,P6,ppath6,ti1)\\
node\_embedding(V1,P1,ti1)\\
node\_embedding(V4,P1,ti1)\\
node\_embedding(V2,P2,ti1)\\
node\_embedding(V3,P3,ti1)\\
node\_embedding(V5,P4,ti1)\\
node\_embedding(V6,P5,ti1)\\
link\_embedding(vlink1,plink1,ti1)\\
link\_embedding(vlink2,plink2,ti1)\\
link\_embedding(vlink3,plink1,ti1)\\
link\_embedding(vlink3,plink4,ti1)\\
link\_embedding(vlink4,plink5,ti1)
