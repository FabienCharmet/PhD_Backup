\label{sec:security_prop}
In this section, we define the different security properties we want to preserve in the environment during the migration.
We propose several security properties: confidentiality, integrity, availability, authenticity and co-residency.
After providing a formal definition of the security property, we refine it into several properties related to the operation of SDN networks and virtualization. Each property concerns physical elements of the SDN infrastructure, and we provide the equivalent properties for the virtual nodes and links if they exist. 
% For the rest of this section, we note $E_n$ (respectively $E_l$) the underlying resources embedding virtual node $n$ (respectively virtual link $l$).

\subsubsection{Confidentiality}
\label{sec:prop-conf}
Confidentiality is a characteristic that applies to information.
To protect and preserve the confidentiality of information means to ensure that it is not made available or disclosed to unauthorized entities~\cite{ISO/IEC270012013}.
Therefore, in network security, the confidentiality of a message traveling in the network is defined by the ability of a user to read data forwarded by an element of the network.
Formally we can define a predicate $confidentiality(D)$ as follows:

\begin{myformula}
$ confidentiality(D) \Leftrightarrow \forall U,t~reads(U,D,t) \Rightarrow is\_authorized(U,D,t)$
\end{myformula}

\begin{itemize}
\item $reads(U,D,t)$ is true if user $U$ reads data $D$ at time $t$.
\item $is\_data\_authorized(U,D,t)$ is true if user $U$ has the right to read data $D$ at time $t$.
%\item $\overline{reads(U,D,t)}$ is the negation of $reads(U,D,t)$
\end{itemize}
Simply put, if at every time no user reads $D$ or every user reading $D$ is authorized to do so, then the confidentiality of $D$ is preserved.
This predicate doesn't account for potential cryptography used on the data.
Ciphered data and plain data are treated as the same regarding their confidentiality, only the consequences differ and this is not the scope of this model.
From this definition, we can describe several properties directly affecting the physical and virtual topology.


% Give names instead of numbers 1, 2?
% node versus network element.  are they the same?

\textbf{Property 1a:} A network element respects the confidentiality of the data it is carrying if and only if the user accessing the network element and the data is authorized to do so.
%\RW{is the table helpful?  the formula seems clearer to me. rw}\FC{Cleaned}
\newline \textbf{Formal definition:} Let $N$ be a network element, $U$ a user (malicious or not), $D$ the data carried by the network, and $t$ the time.


\begin{myformula}
$node\_confidentiality(N) \Leftrightarrow \forall D,U,t~
confidentiality(D) \wedge is\_carrying(N,D,t) \wedge (accessing(N,U,t) \Rightarrow is\_node\_authorized(N,U,t))$
\end{myformula}

\begin{itemize}
\item $accessing(N,U,t)$ is true if user $U$ accesses network element $N$ at time $t$.
\item $is\_node\_authorized(N,U,t)$ is true if the user $U$ has the right to access network element $N$ at time $t$.
\item $is\_carrying(N,D,t)$ is true if node $N$ is carrying data $D$ at time $t$.
\end{itemize}

\textbf{Property 1b:} A virtual node respects the confidentiality of the data it is carrying if and only if the embedding nodes respect the confidentiality of the data as well.
%\RW{is the table helpful?  the formula seems clearer to me. rw}\FC{Cleaned}
\newline \textbf{Formal definition:} Let $n$ be a virtual node and $E_n$ the physical resources embedding $n$.


\begin{myformula}
$vnode\_confidentiality(n) \Leftrightarrow \bigwedge\limits_{N\in E_n} node\_confidentiality(N) $
\end{myformula}

\textbf{Property 1c:} A virtual link respects the confidentiality of the data it is carrying if and only if the embedding nodes respect the confidentiality of the data as well.
%\RW{is the table helpful?  the formula seems clearer to me. rw}\FC{Cleaned}
\newline \textbf{Formal definition:} Let $l$ be a virtual link and $E_n$ the physical resources embedding $n$.


\begin{myformula}
$vlink\_confidentiality(l) \Leftrightarrow \bigwedge\limits_{N\in E_n} node\_confidentiality(N) $
\end{myformula}

\textbf{Property 2a:} A path preserves confidentiality from end to end if all of its network elements preserve their own confidentiality.
\newline \textbf{Formal definition:} Let $P$ be the path taken by the traffic, $L_N$ be the list of nodes composing $P$.

\begin{myformula}
$path\_confidentiality(P) \Leftrightarrow \noindent  \bigwedge\limits_{n \in L_N}node\_confidentiality(n)$
\end{myformula}

\textbf{Property 2b:} A virtual path preserves confidentiality from end to end if all of its virtual nodes and links preserve their own confidentiality.
\newline \textbf{Formal definition:} Let $P$ be the path taken by the traffic, $L_N$ be the list of virtual nodes composing $P$, and $L_L$ the list of virtual links composing $P$.
% and $L_{s}$ be the list of the links between each node of $P$.
\begin{myformula}
$vpath\_confidentiality(P) \Leftrightarrow \bigwedge\limits_{n \in L_N}vnode\_confidentiality(n) \wedge \bigwedge\limits_{l \in L_L}vlink\_confidentiality(l)$
\end{myformula}

\textbf{Property 3:} A node's configuration remains confidential during a migration if the employed communication channel respects the confidentiality of the communication.
\newline \textbf{Formal definition:} Let $N$ be a node, $conf$ be the configuration to be pushed, $C$ be a SDN controller.
\newline
%\RW{ i proposed a replacement but i still have questions about it:  $confidentiality(conf,N) \Leftrightarrow \exists C,P~ confidentiality(C) \wedge confidentiality(P) \wedge confidentiality(N) \wedge is\_controller\_of(C,N)$}
%\RW{needs quantifiers for C U P N.  also, the formula does not constrain these to have anything to do with conf it seems that to define a confidentiality-preserving migration requires mentioning time.  the network is confidential after the migration if it was confidential before.  }

\begin{myformula}
$config\_confidentiality(conf,N) \Leftrightarrow~\exists~C,P 
~is\_controller\_of(C,N) \wedge is\_path(C,P,N) \wedge  conf\_of\_node(conf,N) \Rightarrow
~path\_confidentiality(P) \wedge  node\_confidentiality(N)
$
\end{myformula}


\begin{itemize}
\item $is\_controller\_of(C,N)$ is true if SDN controller $C$ controls node $N$.  
\item $is\_path(C,P,N)$ is true if path $P$ is the path between $C$ and $N$..
\item $conf\_of\_node(conf, N)$ is true if node $N$ is running configuration $conf$.
\end{itemize}

\subsubsection{Integrity}
\label{sec:prop-int}
 To preserve the integrity of information means to protect the accuracy and completeness of the information and the methods that are used to process and manage it~\cite{ISO/IEC270012013}.
Therefore, accuracy and completeness of information can only be evaluated by using its temporal values.
%\RW{unclear--why "therefore";  what are the temporal values of a piece of information? is that the result of eval at a given time?}\FC{Yes, the integrity of information means that it remains constant through time, unless processed, obviously}

We define two predicates, $d\_integrity(D,ti)$ and $p\_integrity(P,D,N,ti)$, which represent the status of the integrity of the data $D$ itself and the process $P$ used to handle $D$, respectively, over a time interval $ti$. 

Therefore, if a data has not been legitimately processed, its integrity holds if it hasn't been modified in any other way.
\begin{myformula}
$d\_integrity(D,ti) \Leftrightarrow \forall t_1,t_2\in ti ~ \overline{processed(D, ti)} \Rightarrow 
~deval(D,t_1)=deval(D,t_2)]$ 
\end{myformula}
%\RW{I changed the formatting for clarity}

\begin{itemize}
\item $processed(D, ti)$ holds true if $D$ has been legitimately processed during time interval $ti$.
\item $deval(D,t)$ returns the value of $D$ at time $t$
%\RW{Does this mean that it is not defined if D has been legitimately processed?  Or it is true if D has been processed? or it does not hold if D has been processed?}\FC{It is still true if D has been legally processed}

\end{itemize}

\begin{myformula}
$p\_integrity(P, ti) \Leftrightarrow \forall t_1,t_2 \in ti~ \overline{modified(P, ti)}\Rightarrow 
~peval(P,t_1)=peval(P,t_2) $
\end{myformula}

\begin{itemize}
\item $modified(P,ti)$ holds true if process $P$ has been modified legitimately during time interval ti
\item $peval(P,t)$ returns the process $P$ at time $t$
\end{itemize}
%\RW{you use P for paths and processes. and p for a function. what is the output of data? is it the value (eval?)}\FC{I don't understand}\GB{beware: you've been using $P$ for two distinct notions. I think that what RW means it that $p$ evaluates the realization of process $P$ over data $D$. In that sense you could, replace $p$ by the keyword $eval$}


\textbf{Property 1a:} The integrity of a network element is preserved if the data it carries and the process used to process it have their integrity preserved. 
\newline
\textbf{Formal definition:} Let $N$ be a network element, $D$ some data and $P$ a function modeling the process used by to handle data.

\begin{myformula}
$node\_integrity(N) \Leftrightarrow \forall D,ti,\exists P~p\_integrity(P,ti) \wedge d\_integrity(D,ti) \wedge process\_of(N,P,ti)$
\end{myformula}

\begin{itemize}
\item $process\_of(P,N,ti)$ affects process $P$ to node $N$ during time interval ti.
\end{itemize}

\textbf{Property 1b:} The integrity of a virtual node is preserved if and only if the nodes embedding it preserve their integrity as well.
\newline
\textbf{Formal definition:} Let $n$ be a virtual node and $E_n$ the list of nodes embedding $n$.

\begin{myformula}
$vnode\_integrity(n) \Leftrightarrow \bigwedge\limits_{N \in E_n} node\_integrity(n)$
\end{myformula}

\textbf{Property 1c:} The integrity of a virtual link is preserved if and only if the nodes embedding it preserve their integrity as well.
\newline
\textbf{Formal definition:} Let $l$ be a virtual link and $E_l$ the list of nodes embedding $l$.

\begin{myformula}
$vlink\_integrity(n) \Leftrightarrow \bigwedge\limits_{N \in E_l} node\_integrity(n)$
\end{myformula}


\textbf{Property 2a:} A path preserves integrity from end to end if all of its network elements preserve their own integrity.
\newline \textbf{Formal definition:} Let $P$ be a path in the network and $L_N$ be the list of nodes composing $P$.
\newline

\begin{myformula}
$path\_integrity(P) \Leftrightarrow \bigwedge\limits_{n \in L_N}integrity(n) $
\end{myformula}

\textbf{Property 2b:} A virtual path preserves integrity from end to end if all its virtual nodes and links preserve their own integrity.
\newline \textbf{Formal definition:} Let $P$ be a virtual path, $L_N$ be the list of virtual nodes composing $P$ and $L_L$ be the list of virtual links composing $P$ .
\newline

\begin{myformula}
$vpath\_integrity(P) \Leftrightarrow \bigwedge\limits_{n \in L_N}vnode\_integrity(n) \wedge \bigwedge\limits_{l \in L_L}vlink\_integrity(l) $
\end{myformula}


\subsubsection{Availability}
\label{sec:prop-avail}
An asset is available if it is accessible and usable when
needed by an authorized entity~\cite{ISO/IEC270012013}.

We can define the predicate $availability(A)$ which represents the availability of asset $A$.
\newline
Let $A$ be an asset, $U$ a user and $ti$ a time interval.
\newline
%\RW{why would availability be discussed without specifying a time?  it seems unlikely that an asset is always useable.}
%\FC{In the same way we consider confidentiality and integrity, as long availability is preserved we don't need time, but we need to pinpoint the time intervals related to the violation if needed.}

\begin{myformula}
$availability(A) \Leftrightarrow \forall ti~is\_accessible(A,U,ti) \wedge is\_usable(A,U,ti) \Rightarrow is\_asset\_authorized(U,A,ti)$
\end{myformula}

% \RW{why would an asset not be available, if it is useable and the user is authorized?  is it because an authorized path does not exist?}
% \FC{I've added authorization}
\begin{itemize}
\item $is\_accessible(A,U,t)$ is true if user $U$ can contact asset $A$ at time $t$.
\item $is\_usable(A,U,t)$ is true if user $U$ can use asset $A$ at time $t$.
\item $is\_asset\_authorized(U,A,ti)$ is true if user $U$ is allowed to access asset $A$.
\end{itemize}

\textbf{Property 1a:} A network node $N$ is available if it can be reached and it can process incoming packets.
\newline
\textbf{Formal definition:} Let $N$ be a network node, $F$ be the function of $N$, $P$ be a list of packets, and $t$ be a point in time.

\begin{myformula}
$node\_availability(N) \Leftrightarrow \forall t,U~ is\_accessible(N,U,t) \wedge can\_process(N,t) \Rightarrow is\_node\_authorized(U,N,ti)$
\end{myformula}

\begin{itemize}
\item $can\_process(N,t)$ is true if $N$ processes packets normally (\eg no congestion etc.)
\end{itemize} 

\textbf{Property 1b:} A virtual node $n$ is available if the physical nodes embedding it are available.
\newline
\textbf{Formal definition:} Let $n$ be a virtual node and $E_n$ the nodes embedding it.

\begin{myformula}
$vnode\_availability(n) \Leftrightarrow \bigwedge\limits_{N \in E_n} node\_availability(N)$
\end{myformula}

\textbf{Property 1c:} A virtual link $l$ is available if the physical nodes embedding it are available.
\newline
\textbf{Formal definition:} Let $l$ be a virtual link and $E_l$ the nodes embedding it.

\begin{myformula}
$vlink\_availability(n) \Leftrightarrow \bigwedge\limits_{N \in E_l} node\_availability(N)$
\end{myformula}

\textbf{Property 2a:} A path is available from end to end if all of its nodes are available.
\newline \textbf{Formal definition:} Let $P$ be the path taken by the traffic, $L_N$ be the list of nodes composing $P$ and $t$ a point in time. 

\begin{myformula}
$path\_availability(P)\Leftrightarrow \forall t, \bigwedge\limits_{n \in L_N}availability(n,t)$
\end{myformula}


\textbf{Property 2b:} A virtual path is available from end to end if all of its virtual nodes and virtual links are available as well.
\newline \textbf{Formal definition:} Let $P$ be a virtual path, $L_N$ be the list of virtual nodes composing $P$ and $L_L$ the list of the virtual links composing $P$.

\begin{myformula}
$vpath\_availability(P)\Leftrightarrow \bigwedge\limits_{n \in L_N}vnode\_availability(n) \wedge \bigwedge\limits_{l \in L_L}vlink\_availability(l)$
\end{myformula}

\subsubsection{Authenticity}
\label{sec:prop-auth}
Authenticity is a property or characteristic of an entity.
An entity is authentic if it is what it claims to be~\cite{ISO/IEC270012013}.

Let $E$ be an entity, $R$ a role for $E$ and $V$ the verification authority.
The authenticity of an entity $E$ can be defined as follow:

\begin{myformula}
$ authenticity(E) \Leftrightarrow \forall t, claim\_role(E,t) \wedge has\_role(E,V,t)$
\end{myformula}
\begin{itemize}
\item claim\_role(E,t) returns the role R of E at time t
\item has\_role(E,V,t) the role R of E is returned by V at time t
\end{itemize}
% \RW{why is authenticity defined without a time argument but claim\-role does have a time argument?}
% \FC{Similarly to confidentiality, authenticity is a binary property that cannot be restored}\GB{what do you mean by ``cannot be restored''?}
% \FC{For authenticity, once the user lied about who he is and we found out, you cannot say "I will still allow you to user our network."}

\textbf{Property 1:} The authenticity of a migration request is verified if the topology migrated belongs to the requesting user, and the user is authentic.
\newline
\textbf{Formal definition: } Let $U$ be a user, $V$ a virtual topology, $r$ the migration request and $N$ the network hypervisor.

\begin{myformula}
$request\_authenticity(r,U) \Leftrightarrow \exists U,~authenticity(U) \wedge belongs\_to(r_V,U)$
\end{myformula}
%\RW{there is no quantifier for V and it is not a parameter.}
%\FC{V is a parameter inside request r²}
\begin{itemize}
\item $authenticity(U)$ is true if $U$ is a tenant known by $N$
\item $r_V$ is the topology concerned by request $r$
\item $belongs\_to(r_V,U)$ is true if the topology $V$ belongs to user $U$
\end{itemize}
%\RW{does it matter what r is in this definition?}
%\FC{r is just the request, it's not under the scope for examination}


\subsubsection{Co-Residency}
\label{sec:prop-cores}
Two virtual topologies are co-residing together if they are embedded on a subset of the same physical resources at the same time.
We define co-residency with the following predicates:
Let $V_1$ and $V_2$ be two virtual topologies belonging to different users, $P_1$ and $P_2$ two sets of physical resources and let $ti$ be a time interval.
\begin{myformula}
$coresidency(V_1,V_2,ti) \Leftrightarrow \exists t_1,t_2\subseteq~ti,P_1,P_2$ $is\_embedding(P_1,V_1,t_1) \wedge is\_embedding(P_2,V_2,t_2) \wedge P_1 \cap P_2 \neq \{\emptyset\} \wedge t_1 \cap t_2  \neq \emptyset$
\end{myformula}

\begin{itemize}
    \item $is\_embedding(P_1,V_1,t_1)$ is true if the physical set of resources $P_1$ embeds the virtual network $V_1$ at time $t_1$.
\end{itemize}

Co-residency is a property slightly different than the others because it is not directly related to security.
However, as shown in the state of the art, co-residency is an important metric for the success of an attack.
Indeed, side channel attacks on Virtual Machines often relies on the co-residency of both the victim's and the attacker's VMs.



\subsubsection{Mapping formal predicates with SDN events}
\label{sec:extending-model}
In the previous sections we have described the different properties we want to observe inside the SDN infrastructure.
We present a potential mapping between the formal predicates and the corresponding SDN events or situations.


\paragraph{Confidentiality}
The confidentiality section introduces the following predicates:
\begin{itemize}
\item reads(U,D)
\newline
This predicate is true if either a monitoring tool accesses traffic or if the traffic is forwarded to a virtual machine. (Note the VM may not be legitimate.)

\item is\_data\_authorized(D,U)
\newline
This predicate is true if user U is the intended recipient of data D (\eg destination IP).
D can be inspected with monitoring tools to verify the destination IP field.
\item is\_node\_authorized(N,U)
\newline
This predicate is true if user U is authorized to access the physical node.
This is a typical use of Access Control.
\item access(N,U)
\newline
This predicate is true if U logs in N using a CLI or use remote access to interact with N.
The logs of N can be exploited to verify the accesses made.
\item $is\_carrying(N,D,t)$ can be verified by checking the flow rules composing the path taken by $D$.
\end{itemize}

\paragraph{Integrity}
The integrity section introduces the following predicates:
\begin{itemize}

\item processed(D,ti)
\newline
If D has reached its final destination before ti then the predicate holds true.
Otherwise, the forwarding process performed on D during ti can be verified by checking the related flow rules, and D can be analyzed to see if it matches the expected results. (\ie no extra illegitimate manipulation have been performed).

\item modified(P,ti)
\newline
The evolution of P implies legitimate modification of the corresponding flow rules, which can be monitored by the hypervisor.

\item deval(D,t)
\newline
This is done by reading the value of D, using for instance a monitoring tool.

\item peval(P,t)
\newline
The process P can be monitored by querying the related flow rules

\item proccess\_of(P,N,ti)
\newline
The flow rules of node N can be queried to check if they matches process P.

\item \FC{Should I reintroduce is\_node\_authorized ?}
\end{itemize}

\paragraph{Availability}
The availability section introduces the following predicates:
\begin{itemize}
\item is\_accessible(N,t)
\\ This predicate is evaluated by regularly sending an Openflow ECHO request to the node.
\\ SDN controllers already send such requests.

\item can\_process(N,t)
\\ OpenFlow switches provides statistics and primitives can show the amount of CPU power used.

\item \FC{Should I reintroduce is\_node\_authorized ?}
\end{itemize}

\paragraph{Authenticity}
The authenticity section introduces the following predicates:
\begin{itemize}

\item claim\_role(E)
\item has\_role(E,V)
\\
Those two predicates are verified by the network hypervisor upon receiving a migration requests.
The DPAC will provide an interface to identify tenants to let them interact with the hypervisor.
\item belongs\_to($r_V$,U)
\\
The network hypervisor keeps a mapping between virtual topologies and their owner, hence it is trivial to verify this predicate.
\end{itemize}
 
 
\paragraph{Co-Residency}
The Co-Residency section introduces the following predicate:

\begin{itemize}
    \item is\_embedding($P_1,V_1,t1$) is verified by the hypervisor. Alternatively, checking the corresponding flow rules of the nodes in $P_1$ verifies this predicate.
\end{itemize}