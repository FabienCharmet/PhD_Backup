This work aims at providing a security verification framework to determine if the migration process encountered a security breach.
We decompose this objective into three different tasks: designing a formal model, linking the formal model with networking events and the specificities of the SDN environment and finally computing the formal proof for whether or not the security properties have been violated during the migration.
 
We propose a formal model and definitions for the following security properties: confidentiality, integrity, availability 
% , authenticity 
and co-residency, the latter being a property specifically related to the security of the virtualization infrastructure. First, the model must describe the physical and virtual environments, how is a virtual network embedded on a physical substrate and how one virtual element, node or link, can be mapped with several physical nodes (\ie the one-to-many mapping).
Then the model should provide a basic definition of the security property and then refine it through a serie of properties that are specific to SDN virtualization and the migration.

The next task is to link the formal predicates with networking or SDN related events that will serve as the basis to construct the formal trace of the migration process.
Monitoring tools can then be used to detect specific events and interactions in the network. Such events will then be converted into the corresponding formal predicates and will constitute an execution trace from the formal point of view.
We will implement a simulator for such purposes.

Finally, the formal trace generated in the previous task will be used by a theorem prover to construct a formal proof of the preservation, or the violation, of the security properties associated to the migration process. The theorem prover has to provide temporal reasoning facility as the migration process is seen as a chronological sequence of events that may ultimately lead to a security breach.

% Finally, we present how the verification of the security is performed, using the SNARK~\cite{snark-Stickel2000} model checker, and we illustrate the work with a use case.Finally, we present how the verification of the security is performed, using the SNARK~\cite{snark-Stickel2000} model checker, and we illustrate the work with a use case.



