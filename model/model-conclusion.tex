Our analysis of the state of the art has outlined a lack of consideration for the security study of the migration process of a Virtual Network.
One logical explanation is that the migration process is rarely implemented in a network hypervisor because it corresponds to specific operational needs. Additionally, there is only a limited number of commercial solutions for SDN-based hypervisors which may need to migrate Virtual Networks. We believe that those two factors explain the limited study of the migration process.


We aimed at modeling the migration process of Virtual Networks, and the corresponding security properties.
A formal model is useful to define the scope of the study as well as giving a clear definition of the different notions pertaining to the problem considered.
This approach is even more relevant when the problem has never been investigated before, and therefore has no established grounds to base the study on.
In this chapter, we have presented a formal model describing a virtualization infrastructure, how Virtual Networks are migrated  and a set of security properties related to the migration process. These properties were defined by considering them in a very general way, and then refining them more precisely into an SDN context. We have evaluated the formal model with real life use cases, considering attacks existing on SDN infrastructures, and using a simulator to generate the execution trace. Finally, we used the theorem prover SNARK to compute a formal proof of the security violation during the migration. Our results demonstrate the feasibility of formally proving the preservation or the violation of the security properties, based on the analysis of the execution trace.

Despite the fact that the co-residency is not a security property \textit{per se}, we define it to illustrate the evolution of the attack surface that can be exploited to compromise the operation of the migration process, as it is done for traditional hypervisors~\cite{stalling-atya2017,malicious-atya2017}.
% Abnormal behaviours could be detected using co-residency and attackers could be exposed for trying to be located on the same physical resources as their victim. \CK{C'est un peu etrange de lire un paragraphe entier sur la co-residency alors que c'est tres mineur dans ton formalisme et pas vraiment exploite}\FC{Genre un truc plus court ou pas du tout ?}

The use of a formal approach to verify real life problems inevitably raises a question about the realism between the model and the system it describes.
In our case, the detection of security violations relies on the assumption that we can detect and characterize every relevant network event and that the trace generated from the migration is complete. However, this assumption is too strong for a real life implementation of our model. Current IDS solutions do not process large amounts of events, because they are manually examined by an operator.
Additionally, the attack may not be detected in a voluminous traffic.
% Therefore, we propose to alleviate this assumption of perfect monitoring.
% As it is, \GB{where is the end of the sentence. I found the challenge awkwardly posited...}.

In the next chapter, we propose to alleviate the assumption that the monitoring is perfect by proposing a resource allocation model that will optimize the coverage of the security inside the infrastructure.