\subsubsection{Formal theory about theorem proving}
\label{sec:proving-theory}
We present in this section specific aspects of first order logic theorem proving that will be used later on in this thesis. We will give a brief overview of the first order logic theory and we will then present the basics of theorem proving and several verification mechanisms.

\paragraph{First order logic}
The first order logic is a formal system describing a theory using declarative propositions, predicates and quantification.

A declarative proposition, also referred to as a clause, is a set of terms  that may be connected together using logical operators.
We illustrate this with the following example:

$P \Rightarrow Q$ means "$P$ implies $Q$``. $P$ and $Q$ are the terms of the clause and $\Rightarrow$ is the logical operator representing implication.

The first order logic admits the following logical operators:

\begin{itemize}
    \item $\forall$: The forall operator, $\forall x\in S$ means for all elements $x$ in set $S$.
    \item $\exists$: The exist operator, $\exists x\in S$ means there exist at least one $x$ in set $S$.
    \item $\neg$: The negation operator, $\neg A$ means then logical negation of A, often noted $\overline{A}$.
    \item $\wedge$: The conjunction operator, $A \wedge B$ means the logical arithmetic A AND B.
    \item $\vee$: The disjunction operator, $A \vee B$ means the logical arithmetic A OR B.
    \item brackets $( )$: Brackets are used to order the resolution of the logical proposition, similarly to traditional arithmetic.
    \item $\Rightarrow$: The implication operator, $A \Rightarrow B$ means A implies B.
    \item $\Leftrightarrow$: The bidirectional implication operator, $A \Leftrightarrow B$ means "A implies B`` AND "B implies A``.
\end{itemize}

Predicates are functions taking a set of terms (variables) as input and returning a binary value, True or False, as an output.

For instance, we can define $P(A,B,C) = A \wedge (B \vee C)$, which would yield for the following example: $P(True,False,True)=True$.

\paragraph{First order logic Theorem Proving}\textbf{\\}
The purpose of theorem proving is to determine if a set of clauses is satisfiable or unsatisfiable.
A satisfiable clause that admits a solution that makes it true.
An unsatisfiable clause is a clause that cannot be true no matter the values of its terms.
The theorem prover will apply inference rules on the original set of clauses and try to reach another set of clauses where the satisfiability can be determined.

We will now present some rules of inference used for first order logic theorem proving.
A rule of inference is a rule defining the deduction process, it takes a set of propositions as input and returns a proposition which is a conclusion drawn from the input propositions.

We use $\vdash$ to represent the result of the deduction from a set of proposition.

\paragraph{Modus ponens and modus tollens}\textbf{\\}
Modus ponens is a rule of inference using the affirmation in an implication to deduce the conclusion.
Considering the simple example given previously, $P \Rightarrow Q$, if we assert that $P$ is true, then we use modus ponens to conclude that $Q$ is true as well. This is summarized as "If P implies Q and P is true, then Q must be true" : $P \Rightarrow Q, P \vdash Q$.

Similarly Modus tollens uses the negation in the consequence of an implication to deduce the conclusion.
Considering $P \Rightarrow Q$, asserting that $Q$ if false, then $P$ must be false as well: $P \Rightarrow Q, \neg Q \vdash \neg P$.

\paragraph{Resolution principle}\textbf{\\}
The resolution principle, shortened to resolution, is an inference rule satisfying the completeness property. Completeness implies that applying the resolution to an unsatisfiable clause will always yield false. 

For instance, $P \wedge \neg P$ is obviously unsatisfiable, \ie it is always False.

The resolution principle uses substitution in a set of clauses to reach a conclusion.
We consider the simple example presented in~\cite{snark-Stickel2000} and refer the reader to~\cite{symbolic-proof} for additional reading, more specifically on variable unification.
We also refer the reader to the Method of Davis and Putnam and more specifically on the One-Literal Rule.

Consider the following clauses: $\neg P \vee Q$ and $P \vee R$. The resolution principle returns $Q \vee R$ as a resolvent of the previous propositions.


\paragraph{Paramodulation}\textbf{\\}
The paramodulation is an inference rule used to exploit the equality existing between terms and the occurrence of term in a proposition. The paramodulation can be seen as \textit{substitution in case of equality}.

Consider the proposition $P(x) = true$ and the quality $x=y$. Therefore, we can infer that $P(y)$ will also yield True.

Consider now the two propositions: $P(x) \vee Q(y)$, $(x = y) \vee R(y)$, paramodulation deduces the proposition $P(y) \vee Q(y) \vee R(y)$.\\
$P(x) \vee Q(y)$, $(x = y) \vee R(y) \vdash P(y) \vee Q(y) \vee R(y)$

\paragraph{Rewriting rules}\textbf{\\}
Rewriting rules are used to replace an expression by another in a set of propositions.
This is done to simplify the search space when applying inference rules~\cite{snark-Stickel2000}.

We illustrate this with a simple example: $P \Leftrightarrow Q \wedge R, P \vee T$ is rewritten as $Q \wedge R \vee T$. 

The rewriting rule may be condition to replace the left hand side of an equivalence by the other, but may also apply substitutions to terms in the proposition only on one side of the proposition.
We also refer the reader to~\cite{snark-Stickel2000,symbolic-proof} for further explanations.





\subsubsection{SNARK: A first order temporal logic theorem prover}
\label{sec:proof-detects-violation}
%Part of this effort is devoted to using automatic reasoning to establish that a network enforces confidentiality properties or to detect that such properties can be violated.
In section~\ref{sec:proving-theory}, we have described some aspects of the theory of first order logic theorem proving.
In this section we  present a first-order theorem prover that will be used to detect the violation of the security properties.
We use the Stanford Research Institute's open-source first-order reasoner SNARK~\cite{snark-Stickel2000}.

SNARK is fully automatic and uses machine-oriented inference rules, such as \textit{resolution} for general-purpose reasoning and \textit{paramodulation} and \textit{rewriting} for reasoning about equality.
It has special facilities
%\GB{is listing all features supported by SNARK relevant in the paper?} 
for accelerated reasoning about space and time; a sort mechanism for internalizing the reasoning about sorts or types; and an \textit{answer extraction} mechanism for constructing an answer to a query from a completed proof. 
Although SNARK is ideally suited for our purpose and has been successfully used in other applications (\eg~\cite{AICPub2006:2015}), any other theorem prover with comparable features could be used.

SNARK is a refutation procedure, \ie it looks for inconsistencies in a set of logical sentences. It begins with a theory, a set of axioms, \ie logical sentences that specify the basic properties of the subject domain. 
When trying to prove that a conjecture holds in the theory, it negates the conjecture, adds it to the set of axioms of the theory, and tries to find a contradiction in the resulting set. 
For this purpose, it will apply inference rules, which deduce new sentences, and adds those to the set. The process continues until no further inferences are possible or until a contradiction is obtained \-—-- this shows that the negated conjecture is inconsistent with the axioms of the theory and, hence, that the conjecture itself is valid in the theory. 
Once the conjecture is proven, it is regarded as a theorem of the theory; before that, its validity is in question.
The answer extraction mechanism, developed for program synthesis applications~\cite{Manna:1980:DAP:357084.357090}, associates with each sentence in the set an answer term, such that if the sentence can be falsified, the corresponding instance of the answer term will satisfy the original query. 
When a contradiction is obtained, any instance of the contradictory sentence is false, and hence the corresponding answer term will always  be an answer to the query. 
Typically, there are many proofs of a given theorem, each leading to a possibly different answer. 
These can be assembled and presented to the user.

The order in which inference rules are applied depends on a set of heuristic principles which are under the control of the developer of the theory. 
Weights can be assigned to symbols, which cause SNARK to focus attention on “lighter” sentences in its set. 
Also, an ordering is applied to symbols, which cause SNARK to focus on particular parts of a sentence. 
Typically, symbols will be replaced by other symbols that are smaller in the ordering. 
Since first-order theorem proving is undecidable and the search can go on forever, we give SNARK a time limit on proving a given theorem. 
This is set empirically \-—-- for the examples considered here, most of the theorems were proven in less than a second. The use cases took no more than 30 seconds.
SNARK will automatically search for a series of proofs of a given conjecture, each yielding a different answer.
In general, there may be an infinity of answers to a query, but we rely on the time limit mechanism to cut off the search.

%\GB{most of the contents before describe how SNARK works in rather technical terms. Although interesting, it may be out of the scope of the paper to explain in such details how SNARK works. It should be summarized to highlight what are tehe most important features of SNARK formalizing and solving our migration problem. To that end, the illupstrative example below is what is expected from this paper}

\subsubsection{Proving a simple violation with SNARK}
\label{sec:find-violation}
The operation and formalism of SNARK will be illustrated in the context of a simple example.
We are given an axiomatic theory of the domain of network confidentiality, and we use SNARK to detect a confidentiality violation during the migration of a virtual topology.
This example makes a simple assumption regarding the possibility of reading a data in the network. If the data is carried by a node during a certain time interval, then it can be read via the nodes connected to it, directly or via a path of nodes, during a subset of this time interval.
This assumption alleviate the unnecessary burden of modeling the actual routing performed by network nodes and will instead focus on the configuration installed in the nodes.
Rather than presenting the axioms of the theory in advance, we will introduce them as they are required in the proof.

Note that SNARK adopts a LISP-like notation for expressions; thus, we write \verb'(P A B)' rather than \verb'P(A, B)'. 
Variables with "\verb'?'" prefixes indicate entities that we must find as the proof is under way.
Variables in a sentence are ``dummies'' \---- they may be systematically renamed without changing the meaning of the sentence. 
The notation \verb'?t.t-int' is a variable \verb'?t' of sort \verb't-int', \ie a time interval. 
Other variables are of sort \verb'user', \verb'data', and \verb'path'.
As the proof is under way, these variables will become instantiated with concrete terms of compatible sorts, \eg \verb'?u.user' with \verb'user-1'. 
These instantiations will be reflected in the corresponding answer term.

To find violations, we propose the \textit{data confidentiality violation conjecture}

\begin{lstlisting}[numbers=none] 
(exists (?u.user ?d.data ?p.path ?t.t-int)
  (data-confidentiality-violation 
     ?u.user ?d.data ?p.path 
     ?t.t-int))
     :answer '(ans ?u.user ?d.data ?p.path ?t.t-int)
\end{lstlisting}

In other words, we are establishing the existence of  a user, some data, a path of nodes, and a time interval that exhibits a violation of confidentiality. 
Once we find entities that exhibit the violation, they will compose the answer term that SNARK will extract from the proof:  
  \begin{lstlisting}[numbers=none] 
     (ans ?u.user ?d.data ?p.path ?t.t-int)
  \end{lstlisting}


A data confidentiality violation is defined in part by the following axiom:   
\begin{lstlisting}[]                                               
(implied-by
  (data-path-violation 
     ?u.user ?d.data ?p.path ?t.t-int)
  (and
    (= ?t.t-int
       (intersection
    	  (time-i ?r1.t-pt ?r2.t-pt)
    	  (time-i ?s1.t-pt ?s2.t-pt)))
    (reads-via-path
       ?u.user ?d.data ?p.path 
       (time-i ?r1.t-pt ?r2.t-pt))
    (not (is_authorized
       ?u.user ?d.data
       (time-i ?s1.t-pt ?s2.t-pt)))
    (non-empty ?t.t-int)))
\end{lstlisting}

In other words, the confidentiality violation can occur for a piece of data via a path of nodes during a time interval \verb'?t' if
%\GB{if all these conditions should hold together, I suggest to use the paraenum environment instead}
\begin{itemize}
\item the user can read the data via the path during a time interval \verb'?r' (lines 9-11)
\item the user is not authorized to read the data during a time interval \verb'?s' (lines 12-14)
\item the time interval \verb'?t' is the intersection of \verb'?r' and \verb'?s' (lines 6-8)
\item the time interval \verb'?t' is not empty (line 15)
\end{itemize}

Here  \verb't-pt' is the sort of time points, and \verb'(time-i ?t1.t-pt ?t2.t-pt)' is the time interval between the two time points \verb'?t1.t-pt' and \verb'?t2.t-pt'. If a user can read some data during one time interval and is not authorized to read that data during another time interval, a confidentiality violation can occur during the intersection of those two intervals, provided that the intersection is not empty.

Applying the resolution rule to the negated conjecture and the definition of a data confidentiality violation, and then simplifying, yields the following \textit{current proof step}, which is added to the set of consequences:

\begin{lstlisting}[] 
   (or 
    (not 
      (= ?t.t-int
         (time-i 
          (max-time ?r1.t-pt ?s1.t-pt)
          (min-time ?r2.t-pt ?s2.t-pt))))
    (not (reads-via-path
           ?u.user ?d.data ?p.path
           (time-i ?r1.t-pt ?r2.t-pt)))
    (is_authorized ?u.user ?d.data
       (time-i ?s1.t-pt ?s2.t-pt))
    (not (= ?t.t-int
            (time-i ?t1.t-pt ?t2.t-pt)))
    (not-before ?t1.t-pt ?t2.t-pt))
\end{lstlisting}
with associated answer term \verb' (ans ?u.user ?d.data ?p.path ?t.t-int)'.
This may be regarded as the negation of a goal to be established (line 2).
% ; conditions in the sentence are implicitly negated.
Thus, we are looking for a user who can read the data via a path of nodes but who is not authorized to do so (lines 7-11).
We establish that the intersection is not empty by showing that its endpoints are in temporal order, \ie the negation of \verb' (not-before ?t1.t-pt ?t2.t-pt)' (line 14).

Note that the intersection of the two intervals has been simplified to a single time-interval whose start point is the maximum of the two start-points and whose finish point is the minimum of the two finish points.  
This has been accomplished by rewriting the term according to an axiom of the theory:

\begin{lstlisting}[numbers=none]
 (= (intersection
          (time-i ?r1.t-pt ?r2.t-pt)
          (time-i ?s1.t-pt ?s2.t-pt))
       (time-i
          (max-time ?r1.t-pt ?s1.t-pt)
          (min-time ?r2.t-pt ?s2.t-pt)))
\end{lstlisting}


The relation \verb'reads-via-path' is defined in part by the axiom

\begin{lstlisting}
(implied-by
  (reads-via-path
    ?u.user ?d.data ?p.path ?t.t-int)
  (and
    (data-at-node 
       ?d.data ?n1.node 
       (time-i ?r1.t-pt ?r2.t-pt))
   (reads-at-node 
       ?u.user ?d.data ?n2.node 
       (time-i ?s1.t-pt ?s2.t-pt))
   (path-link 
       ?n1.node ?n2.node ?p.path 
       (time-i ?u1.t-pt ?u2.t-pt))   
    (= ?t.t-int
       (intersection
	     (time-i ?r1.t-pt ?r2.t-pt)
	     (intersection
           (time-i ?s1.t-pt ?s2.t-pt)
           (time-i ?u1.t-pt ?u2.t-pt))))))
\end{lstlisting}

The relation \verb'(path-link ?n1.node ?n2.node ?p.path ?t.t-int)' holds if, at time interval \verb'?t.t-int', there is a connected path \verb'?p.path' linking nodes \verb'?n1' and \verb'?n2'. 
The axiom says that a user can read a piece of data via a path at a time interval \verb'?t.t-int' if
%\GB{if all these conditions should hold, I suggest to use the paraenum environment}
\begin{itemize}
\item the data is at one node during a time interval (lines 5-7)
\item the user reads the data at a second node during a second time interval (lines 8-10)
\item the two nodes are linked via a path of nodes during a third time interval (lines 11-13)
\item the time interval \verb'?t-int' is the intersection of the three time intervals (14-19)
\end{itemize}

The relation \verb'(path-link ?n1.node ?n2.node ?p.path ?t.t-int)' is defined as follows:

\begin{lstlisting}
     (assert
   '(implied-by
     (path-link ?x.node ?z.node
      (cons ?x.node (cons ?y.node ?yz.list))
      (make-interval ?t1.time-point ?t2.time-point))
     (and
      (=
       (make-interval ?t1.time-point ?t2.time-point)
       (intersect-ii
	(make-interval ?r1.time-point ?r2.time-point)
	(make-interval ?s1.time-point ?s2.time-point)))
      (setlink0 ?x.node ?y.node
       (make-interval ?r1.time-point ?r2.time-point))
      (path-link ?y.node ?z.node
       (cons ?y.node ?yz.list)
       (make-interval ?s1.time-point ?s2.time-point)))
     ))
\end{lstlisting}

We establish that nodes \verb'?x.node ?z.node' (line 3) are forming a path if they are indirectly connected via a third node \verb'?y.node'. We assert the existence of a direct connection for nodes \verb'?x.node' and \verb'?z.node' using \verb'setlink0' assertion (lines 12-13) and recursively construct the rest of the path towards \verb'?z.node' by using the relation \verb'path-link' again (lines 14-16).

We are given axioms that assert that \verb'node-a' is linked to \verb'node-b', and \verb'node-b' is linked to \verb'node-c', during the time interval \verb'(time-i time-b time-c)'. We are also given that \verb'data-1' is at \verb'node-a' during the same time interval 
\verb'(time-i time-b time-c)' and that \verb'user-2' reads \verb'data-1' at \verb'node-c' during the larger time interval \verb'(time-i time-a time-c)'.  

Using these axioms in the proof causes the variable in the proof step, including the answer term, to be instantiated.  
For instance, when we use the fact that \verb'user-2' reads \verb'data-1' at \verb'node-c', the variable \verb'?u.user' is replaced by \verb'user-2', the variable \verb'?d.data' is replaced by \verb'data-1', and the variable \verb'?n2.node' is replaced by \verb'node-2'.  
The time interval \verb'?t-int' is taken to be \verb'(time-i time-b time-c)', the intersection of all the time intervals involved. The fact that this interval is non-empty is given; in general such inferences invoke SNARK's temporal reasoning facility. 

When the proof is complete, the answer-extraction mechanism extracts the answer:
\begin{lstlisting}[numbers=none]
   (ans user-2 data-1 
        (path node-a node-b node-c)
        (time-i time-b time-c)))).
\end{lstlisting} 
This tells us that a violation can occur if \verb'user-2' reads \verb'data-1' along the path \verb'(note-a node-b node-c)' during the time-interval \verb'(time-i time-b time-c)'. 
% The path is \verb'(path node-a node-b node-c)'. 
This is possible because \verb'data-1' is at \verb'node-a' during the larger time interval 
\verb'(time-i time-a time-c)'.

% It is also possible to use SNARK to establish that confidentiality is maintained in a particular virtual network.  
% For instance, if we pose the \textit{confidentiality list conjecture}
% \begin{verbatim}
%  (and
%   (confidentiality-l (list data-1 data-2) ?t.time-i)    
%   (non-empty ?t.time-i))
% \end{verbatim}
% with answer term \verb'(ans ?t.time-i)', we are asking to find a time interval 
% \verb'?t.time-i' during which the confidentiality of \verb'data-1' and \verb'data-2' is maintained in the network.
% We define the \verb'confidentiality-l' conjecture with:
% \begin{verbatim}
%       (assert
%   '(iff
%      (confidentiality-l (cons ?d.data ?p.path) ?t.time-interval)
%      (and
%       (data-confidentiality ?d.data ?t.time-interval)
%       (confidentiality-l ?p.path ?t.time-interval)))
%   :sequential :uninherited
%   :name 'confidentialiy-for-non-empty-list)
% \end{verbatim}