\subsubsection{Summary}
We have presented a reference based on the existing network hypervisors using the SDN paradigm.
Table~\ref{tab:comparison-refarchi} summarizes if and how each hypervisor implements the different modules of the reference architecture.
Both DPAC and CPAC are implemented in every hypervisors, as they are essential for a basic support of network virtualization.
Openflow is the main implementation for the DPAC as it is a standard supported by the industry and is widely studied.
Providing a mapping between physical and virtual resources in the DPAC has become a main goal since arbitrary topologies often come with flowspace virtualization, thus giving tenants a lot of flexibility.
In a similar way, the VNEC greatly simplifies the administration of virtual networks and thus has been integrated in early prototypes.
While FlowVisor provides CPU and bandwidth (BW) isolation, only Sliaces Isolator~\cite{SlicesIsolator-El-Azzab2011} and Double FlowVisor~\cite{DoubleFV-Yin2013} extend this isolation to the interfaces (Int) of a switch as well as the memory (Mem).
We observe that few solutions have implemented an advanced monitoring module (\ie monitoring more than Discovery Protocols Notifications).
In addition to that, network hypervisors do not tackle the migration problem often, leaving a lot of space for resarchers.


