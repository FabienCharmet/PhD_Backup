\subsubsection{Summary}
We have presented a reference based on existing network hypervisors using the SDN paradigm.
Table~\ref{tab:comparison-refarchi} summarizes it and shows how each hypervisor implements the different modules of the reference architecture.
Both DPAC and CPAC are implemented in every hypervisor, as they are essential for a basic support of network virtualization.
The control over the VN exposed by the CPAC is either hypervisor based (HB) or full control (FC).
% Openflow~\cite{Openflow-McKeown2008} is the main implementation for the DPAC as it is a standard supported by the industry and is widely studied.
Maintaining a mapping between physical and virtual resources in the DPAC has become a main goal since arbitrary topologies often come with flowspace virtualization, thus giving tenants a lot of flexibility.
Only Compositional Hypervisor~\cite{CompositionalHypervisor-Jin2014} does not properly virtualize the dataplane and interact with it using OpenFlow~\cite{Openflow-McKeown2008}.
In a similar way, the VNE algorithms greatly simplifies the administration of virtual networks and thus has been integrated since early prototypes.
While FlowVisor provides CPU and bandwidth (BW) isolation, only Slices Isolator~\cite{SlicesIsolator-El-Azzab2011} and Double FlowVisor~\cite{DoubleFV-Yin2013} extend this isolation to the interfaces (Int) of a switch as well as the memory (Mem).
We observe that few solutions have implemented an advanced monitoring module (\ie monitoring more than Discovery Protocols Notifications).
In addition to that, network hypervisors rarely address the migration problem, leaving a lot of work for researchers.


