\subsubsection{Introduction}
We study existing network hypervisors through the prism of virtual network migration.
The scope is restricted to general purpose platforms and equipments to provide the network virtualization service. Therefore, hypervisors using specific technologies such as WiFi, radio frequencies or optical fiber \etc. are out of the scope of this work.

We categorize existing hypervisors using two criteria, how virtual networks are presented to the tenants and whether or not the migration of virtual networks is possible in the considered hypervisor.

\textbf{Abstracting the physical infrastructure\\}
The first criterion used to classify existing hypervisor is how the virtual network is presented to the tenant. We have presented in Section~\ref{def:netvirt} the possible abstractions: slicing, mapping or using a specific API. 

\textbf{Migrating virtual networks\\}
The second criterion we consider is the ability of the network hypervisor to migrate a virtual network in case of a physical failure or an attack on the infrastructure.


