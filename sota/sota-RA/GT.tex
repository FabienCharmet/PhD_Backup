\subsection{Game Theory for Network Security}

Game theory is a mathematical formalism used to represent the interactions between several agents (also known as players).
The model is called a game, in which players have the possibility to choose an action and are rewarded based on their choice and the choice made by the other players. We refer the reader to Appendix~\ref{sec:appendixgt} for more definitions about the formalism.

In this section, we review different games used to solve a resource allocation problem in a security context.

Game Theory has been applied to network security in different ways (e.g., IDS configuration and optimization), as well as, to monitoring resource allocation problems. We will focus on the latter. We summarized the different works in Table~\ref{tab:sota-gt}.
% \GB{the table still does not include information on resolution}.

%  \subsubsection{Static games}
 One of the first works using Game Theory applied to network security is presented by Kodialam~\etal in~\cite{MuraliKodialam2003}, where they model a link sampling for an attack detection problem.
 The aim is to determine the sampling frequency that will optimize the chances of detecting an attack going through the network.
 An attacker launches attacks from a single node and chooses one out of several paths to reach the victim's node.
 If one packet reaches the victim without being detected, the attack is considered successful.
 Both attacker and defender are usually under budget constraints to decide which strategy to implement.
 The formulation is represented as a minmax problem where the attacker attempts to minimize the maximum detection probability of the defender.
 The game is then solved using the max-flow algorithm~\cite{maxflow}, which determines the attacker's strategy as well as the corresponding optimal sampling distribution.
 Otrok~\etal extend this work in~\cite{otrok1,otrok2}.
%  where the model is extended in two ways: either the game is played with imperfect information or the attacker model is augmented.
 In~\cite{MuraliKodialam2003}, the attacker is forced to choose one path to launch his attack, while, in~\cite{otrok1}, the attacker may fragment his attack through several paths in the network.
 Two scenarios are examined: one attacker sending multiple packets and multiple attackers sending one packet.
 In the first case, the attack is deemed successful if the attacker can send a certain amount of packets without being detected.
 In the second case, the attack is a success if every attacker can send one packet to the victim without being detected.
 The game is solved similarly as in~\cite{MuraliKodialam2003}.
 The second extension (\cite{otrok2}) considers a MANET network as the infrastructure.
 A MANET network is composed of several clusters geographically close from each other.
 Each cluster is comprised of several nodes, and regularly one node in the cluster is elected as leader and asked to perform the detection tasks. The authors address several issues arising with this election system, for instance considering that a node might be reluctant to perform intrusion detection.
 Since the real intentions of the nodes in the cluster are not precisely known, a reputation system is considered to determine the amount of trust given to each node.
 Formally speaking, the game is said of imperfect information because it cannot be determined if a leader node has loyally fulfilled its duties and the leader does not know which user is malicious. 
 The game aims at maximizing the detection probability of each link when the probability of selecting each link is fixed. The solution is computed using the Nash Equilibrium.
 Simulation results show that the game performs better than a reference game up until a point where it becomes a tradeoff between energy consumption and false positive rate.
%  \CK{Je suggere d'ajouter au moins une phrase sur la resolution du jeu et les resultats sinon on ne voit pas vraiment l'interet d'avoir parle de cet article}
 
%  \GB{the start of the following paragraph sounds weird: please rewrite}
 An important aspect of network security is to determine which nodes in the infrastructure are the most likely to be targeted. 
%  Agah~\etal~\cite{agah2004} consider this problem in a sensor network, where sensors are grouped into clusters and where one sensor is the clusterhead that will support the detection tasks.
%  The game considered in this situation is usually to let the attacker choose one or more clusterheads to attack, or to do nothing, while the defender will choose which clusterhead to defend.
%  Attacking and defending the infrastructure incur certain costs and rewards that are expressed in the utility functions. \CK{Phrase totalement inutile. Globalement tu ne dis pas grand chose sur cet article. En l'etat je suggererais de supprimer ce paragraphe et de laisser tomber la reference a Agah}
%  \GB{since Chen et al. studied the problem in the traditional network, why not starting by this work in this section?}
%  Chen~\etal~\cite{Chen2009} study the same problem as~\cite{agah2004}, in a traditional network.
 Chen~\etal~\cite{Chen2009} study the target selection problem in a traditional network. 
 Both attacker and defender are under constrained budgets, similarly to the previous works.
 The attacker possesses a certain budget to be spent on attacking the nodes while the defender will allocate resources to protect certain nodes.
 Each node is assigned an asset value, which represents the raw gain obtained by the attacker if he targets this node.
 This raw gain is then impacted with the detection probability and the false positive rate, which determine the final payoffs for both players.
 The game is solved by computing the Nash Equilibrium~\cite{nasheq} to determine the optimal mixed strategies for both players.
 This work has then been extended in~\cite{interdep-ismail2017} to consider the interdependencies of nodes.
 The assumption is that the attacker can target specific nodes because they may grant him some additional privileges to attack other nodes.
 The work concludes with a set of formal results for static security games.
%  They prove that an attacker has no incentive to attack non profitable nodes, and propose resolution methods for a minmax problem. \CK{Enlever cette phrase, trop vague}
 
%  \GB{please elaborate a little more the next paragraph}
 An alternative approach to security games is presented in~\cite{Zhu2009b} where players are given incentives to work together and provide collaborative intrusion detection.
 Each node in the network is constrained by a set of resources and may distribute these resources to other nodes in order to participate to the intrusion detection.
 The game describes an altruistic utility function in which two main criteria are taken into account: the amount of trust between nodes and the satisfaction level for the quality of service provided.
 They prove the existence of a unique Nash Equilibrium and propose an algorithm that converges geometrically fast to the equilibrium.

%  \subsubsection{Stochastic games}
In previous games, the consequences of actions were accounted for in the utility function, where the detection of an attack would impact with a coefficient the gain (resp. loss) for an attacker (resp. a defender).
However, stochastic games allow to model the different possible outcomes of a situation in a more detailed fashion.
For instance, if the attack is successful, the system may transition from a healthy state toward a compromised state, where the attacker is given better rewards in the future.
In this regard, Sallhammar~\etal describe such a game in~\cite{sallhammar2005}.
The system is modeled as a Continuous Time Markov Chain in which the transition probabilities will depend on the attack performed on the system. Similarly to previous work, the attacker may also choose to do nothing.
This work focuses on modelling intentional faults due to attacks and differentiate them from accidental faults using statistical properties such as randomness and independence between faults.
They also consider different types of attacker behaviors where the consequences of being detected may be unknown and/or the attacker may not specifically intend to remain undetected.
The model is then extended in~\cite{Nguyen2009} where the model accounts for interdependencies between nodes to compromise the infrastructure.

Game Theory applied to network security has been surveyed in~\cite{Roy2010,Kiennert2018} where the scope of the survey is extended to other aspects, such as the optimization of an IDS with regard to available detection techniques and countermeasures.
 
%  \GB{I feel Table~\ref{tab:sota-gt} does not feature enough info. E.g., type of game is only static, collaborative, stochastic. what about other properties such zero-sum, Nash equilibrium, the actions used, what algorithm is used for resolution (min-max, max-flow), etc.}
%  \GB{please also indicate authors' names in reference column}
 % Please add the following required packages to your document preamble:
% \usepackage{multirow}
% \usepackage{graphicx}
\begin{table}[ht]
\resizebox{\textwidth}{!}{%
\begin{tabular}{|c|c|c|c|}
\hline
\textbf{Reference}                                                  & \textbf{Type of game} & \textbf{Game Resolution} & \textbf{Summary}                                         \\ \hline
Koadialam~-~\cite{MuraliKodialam2003},~Otrok~-~\cite{otrok1,otrok2} & Static game           & Minmax theorem           & Link sampling allocation for IDS                         \\ \hline
% Agah~-~\cite{agah2004}                                              & Static game           & Minmax theorem           & Monitoring resource allocation in a sensor network       \\ \hline
Chen~-~\cite{Chen2009,interdep-ismail2017}                          & Static game           & Minmax theorem           & Resource allocation  in an infrastructure                \\ \hline
Zhu~-~\cite{Zhu2009b}                                               & Collaborative game    & Iterative algorithm      & Trust management in untrusted environments               \\ \hline
Sallhammar~-~\cite{sallhammar2005}                                  & Stochastic game       & Minmax theorem           & \multirow{2}{*}{Attack modeling for resource allocation} \\ \cline{1-3}
Nguyen~-~\cite{Nguyen2009}                                          & Stochastic game       & Iterative algorithm      &                                                          \\ \hline
\end{tabular}%
}
\caption{Summary of Game Theory for the RA problem}
\label{tab:sota-gt}
\end{table}
