SDN Virtual network migration has proven to be a topic where much work is left to be done from a security perspective.
Existing network hypervisors have yet to make the migration as a default feature, let alone doing so with security considerations.
Formal languages dedicated to security often provide enough expressiveness to model a computer network, and different components related to its security vulnerabilities, and potential existing countermeasures.
However, SDN has yet to be modeled and so are its associated security properties.

We have also emphasized on the security aspects of live Virtual Machine migration, as this process is widely used in the industry to provide uninterrupted service to end users. 
The attack surface of the virtual network migration process is very similar to the one exploited in the different works about Virtual Machines. This attack surface is also augmented by the security issues and challenges introduced by SDN. Nevertheless, the surface of a formal study of the VN migration process is barely scratched, and the security aspects are out of the scope of existing VN migration solutions.

The Game Theory formalism remains difficult to exploit for a realistic use case because of its inadequacy to fit a security problem within its capacities~\cite{Kiennert2018}. \\
Markov Decision Processes have rarely been used for resource allocation problem in a security context. The formal solution a MDP is the set of optimal choices the defender should make when facing every possible situation. Therefore, an MDP can provide an answer to the RA for a security problem, but the solution must be adapted to fit into the requirements and constraints of the security problem.
