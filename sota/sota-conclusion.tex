In this thesis we aim at modeling the security of the virtual network migration process in a SDN environment.
We also aim at making this model to be usable in real life use cases by analyzing how strong assumptions made about the model are, and by proposing alternatives when needed.

SDN Virtual network migration has proven to be a topic where much work is left to be done from a security perspective.
Existing network hypervisors have yet to make the migration a default feature, let alone doing so with security considerations.
Formal languages dedicated to security often provide enough expressiveness to model a computer network, different components related to its security vulnerabilities, and potential existing countermeasures.
However, SDN has rarely been modeled and its associated security properties have never been at all.

We have also emphasized on the security aspects of live Virtual Machine migration, as this process is widely used in the industry to provide uninterrupted service to end users. 
The attack surface of the virtual network migration process is very similar to the one exploited in the different works about Virtual Machines. This attack surface is also augmented by the security issues and challenges introduced by SDN. Nevertheless, the surface of a formal study of the VN migration process is barely scratched, and the security aspects are out of the scope of existing VN migration solutions.

The Game Theory formalism remains difficult to exploit for a realistic use case because of its inadequacy to fit a security problem within its restricted environment~\cite{Kiennert2018}. \\
Markov Decision Processes have rarely been used for resource allocation problem in a security context. The formal solution a MDP is the set of optimal choices the defender should make when facing every possible situation. Therefore, an MDP can provide an answer to the RA for a security problem, but the solution must be adapted to fit into the requirements and constraints of the security problem.

In the following chapters we propose the design of a formal model about the VN migration process and its security. We also describe security properties using a first order logic formalism.
This model will introduce the security of SDN virtualization from a formal point of view, and may serve as the basis for extended definitions of security properties and for the implementation of practical use cases including property verification.

Additionally, we will propose to model the migration process for a Resource Allocation problem, where we want to optimize the distribution of limited monitoring resources over the physical infrastructure.
The application of MDPs to this problem shows that other security problems may be represented with it, and it may not be limited to resource allocation. We expect that complex attacker models may be described and evaluated using this formalism.