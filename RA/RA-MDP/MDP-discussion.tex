\label{sec:mdp-discussion}
The evaluation of our model has outlined several technical and modeling limitations.
The use of budgets as part of the MDP state is a root cause for the complexity in generating the MDP states. Indeed, these budgets act as a sort of ``identifiers'' that make the number of states grow exponentially depending on the budgets. The maximum number of states generated for the MDP is the depth of the recursion function generating the states, multiplied by the number of actions multiplied by the number of transitions per action. This maximum is defined as: $\frac{b_f}{c_f}(2n+1)n = \frac{b_f}{c_f}(2n^2+n)$. This maximum is never reached because there are not $n$ transitions per action since the attacker does not target every node in the infrastructure. Similarly, the minimum number of states is $\frac{b_f}{c_f}(2n+1)$ where there is only one possible transition per action. 

The complexity limitation we encountered was the time it takes to generate all the possible states of the MDP. MDPs can solve use cases with hundreds of actions and hundreds of thousands states, as the value of the states is computed via a multiplication of sparse matrices. As long as the matrices are defined in a reasonable time, solving the MDP does not raise any specific challenge.
Because of the recursive nature of the generation of the MDP states in our problem, the complexity is exponential.
We were not able to generate a use case with $b_f = 50, c_f = 10$, (\ie a depth of 5 recursions).
Similarly, we limited our use case to 6 nodes for the physical topology. Using more nodes was only possible with unrealistic budgets, \eg $b_f = 20, c_f=10$. These budgets means that the maximum size of the migration would be 2 nodes being migrated, with no additional time for the attacker to compromise the migration process. Such numbers were deemed unrealistic and unfit for a proper use case. We suggest to investigate clustering mechanisms to improve the scalability in Chapter~\ref{sec:thesis_conclusion}.

The use of the duration of the migration in the description of each state of the MDP limits the capacities of the attacker. Indeed, the model implies that once the migration is finished, the attacker cannot attack the infrastructure anymore. 
In some numerical applications, we have extended the duration of the migration to one extra iteration ($b_c = 40$). This has given the opportunity for the attacker to extend his attack and reach more nodes.
Because the infrastructure is vulnerable to the attack model described in Section~\ref{sec:attack_model}, it is reasonable to assume that the attacker will keep attacking the infrastructure after the end of the migration. For instance, he may continue to establish his path if he could not complete it prior to the end of the migration. This notion of attack duration is particularly relevant in use cases with more nodes and a migration lasting longer. 
We also outline the fact that the migration process offers a specific attack surface, the communication channel between the hypervisor and the network nodes. Indeed, the configuration rules deployed by the hypervisor may only be intercepted once, as they are sent to the nodes via the communication channel. Once the configuration is installed in the nodes, it requires a different type of attacks (\eg software exploits) to compromise the configuration.

Reflecting on the model we proposed, we raise the following question: ``Which components of an MDP (state, actions, rewards) should be used to represent the behavior of an element external to the system ?". While the formalism does not set any hard limitation, it is important to consider the realism aspect of the modeling with regard to practical scenarios and the threat model used. We represented our attacker in the definition of the states as well as the reward function, since rewards are based on the progression of the attack. 
However, the attacker was modeled with only one type of attack: the configuration modification. Because of the exponential complexity of MDP state generation, each additional parameter makes the generation less and less computationally tractable. Similarly, we could not use a constrained budget for the attack to launch attacks, and instead we used a probability of launching an attack.

We have reviewed Game Theory-based solutions for the Resource Allocation problem.
We considered this formalism as a potential solution to the Resource Allocation for a security problem, but we encountered several limitations with the different types of games (described in Appendix~\ref{sec:appendixgt}). Network security games \FC{Precision: dans la literrature il me semble qu'un jeu de securite est utilise pour du RA (cf les definitions des travaux de Ziad)} usually are two players games, often formulated with the complete information assumption. This assumption is common in network security games, which may not be realistic in real-life scenarios.
While an attacker will expect monitoring counter-measures in the infrastructure, it is not trivial to assume that the attacker knows about the specificity of the counter-measures deployed in the infrastructure, especially if their nature deviates from standard IDS solutions (\eg Moving Target Defense).
Overall,  the realism of a game incurs a cost on the complexity of the solution.
The game will either be realistic but computationally intractable, or the solution can be computed in a reasonable time at the expense of having too many assumptions denaturing the reality of the game.
In our MDP, we have taken the point of view of the defender, and have been able to augment it by representing the attacker interfering with the migration process. 
% Similarly, security games are often considered to be of perfect information, where each player knows the history of the choices made by the other player. However, during an attack, it may take some time to discover that a particular node has been attacked (\eg delay in processing IDS alerts) 


% The use cases we proposed did not distinguish Virtual Networks between one-to-one mapping and one-to-many mapping. While arbitrary topologies are an important feature of a virtualization service, in our model we can easily alleviate this issue because physical nodes embedding virtual resources are part of the migration.


