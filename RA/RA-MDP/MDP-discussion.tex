\label{sec:mdp-discussion}
The evaluation of our model has outlined several technical and modeling limitations:

First of all, the use of budgets as part of the MDP state is a root cause for the complexity in generating the MDP states. Indeed, these budgets act as a sort of ``identifiers" that make the number of states grow exponentially depending on the budgets. The maximum number of states generated for the MDP is the depth of the recursion function generating the states, multiplied by the number of actions multiplied by the number of transitions per action. This maximum is defined as: $\frac{b_f}{c_f}(2n+1)n = \frac{b_f}{c_f}(2n^2+n)$. This maximum is never reached because there are not $n$ transitions per action because the attacker does not target every node in the infrastructure. Similarly, the minimum number of states is $\frac{b_f}{c_f}(2n+1)$ where there is only one possible transition per action. 

Secondly, the use of the duration of the migration in the description of each state of the MDP limits the capacities of the attacker. Indeed, the model implies that once the migration is finished, the attacker cannot attack the infrastructure anymore. 
In some numerical applications, we have extended the duration of the migration to one extra iteration ($b_c = 40$). This has given the opportunity for the attacker to extend his attack and reach more nodes.
Because the infrastructure is vulnerable to the attack model described in Section~\ref{sec:attack_model}, it is reasonable to assume that the attacker will keep attacking the infrastructure after the end of the migration. For instance, he may continue to establish his path if he could not complete it prior to the end of the migration. This notion of attack duration is particularly relevant in use cases with more nodes and a migration lasting longer.

We can generalize the previous point to a global question: ``When modeling the behaviour of an external element to integrate it in the MDP, which component of the MDP should include the modeling?". Simply put, should the external element only be integrated as part of the reward function, include it in the definition of the states or the actions ? While the formalism does not set any hard limitation, it is important to consider the realism aspect of the modeling with regard to practical scenarios and the threat model used. We represented our attacker in the definition of the states as well as the reward function, since rewards are based on the progression of the attack.

% The use cases we proposed did not distinguish virtual networks between one-to-one mapping and one-to-many mapping. While arbitrary topologies are an important feature of a virtualization service, in our model we can easily alleviate this issue because physical nodes embedding virtual resources are part of the migration.


