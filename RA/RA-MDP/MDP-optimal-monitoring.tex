When solving a problem using an MDP, the solution is a dynamic proposition to choose actions as the system evolves.
However, from a technical aspect, the defender needs to have the nodes already monitoring the infrastructure before starting the migration process.
It becomes necessary to translate the dynamic answer of the MDP into a static \textit{a priori} deployment.
After determining the individual importance of each node, we propose to determine the optimal set of monitoring nodes.

The main difference is that each node was evaluated based on all the possible budget combinations, whereas what is defined here is a particular answer for a specific budget.
For each budget, the maximum reward is $\frac{b_f}{c_f} \sum\limits_{i \in \textbf{N}}V_i(1-\alpha) $ which corresponds to the corner case where the attacker never launched an attack, and we can evaluate the efficiency of the  monitoring nodes thanks to the associated reward.
Even if a particular monitoring set achieves close to the maximum reward, it is also because this monitoring set is the optimal solution to a specific subset of all possible attacks.
We propose to determine the optimal monitoring state for each budget by weighting the reward they achieve with their occupation of the total solution space.

% We note $S_{\text{abs}}$ the set of absorbing states, 
We note $S^{\text{Mo}}_{\text{abs}}$ the set of absorbing states with a common  monitoring set $Mo$, $\rho(Mo)$ the percentage of presence of set $Mo$ in the solution space and $R(Mo)$ the reward of monitoring set $Mo$.
% \begin{equation}
%     R(Mo) = \sum\limits_{s \in S_{Mo}^{abs}}\rho(Mo^s)R(Mo^s)
% \end{equation}
Then we propose to choose the optimal monitoring set $Mo^*$ with:
\begin{equation}
    Mo^* = \argmax\limits_{\text{Mo}} \left \{\sum\limits_{s \in S^{\text{Mo}}_{\text{abs}}}\rho(Mo^s)R(Mo^s) \right \}
\end{equation}

The optimal monitoring set is determined by choosing the set giving the maximum weighted reward.
Results for all scenarios are summarized in Table~\ref{tab:mdp-usecase-optiset}.

% Please add the following required packages to your document preamble:
% \usepackage{multirow}
% \usepackage{graphicx}
\begin{table}[]
\resizebox{\textwidth}{!}{%
\begin{tabular}{ccccccc}
\multicolumn{3}{c}{\textbf{Scenario a)}}                                                                     &                       & \multicolumn{3}{c}{\textbf{Scenario b)}}                                                                    \\ \cline{1-3} \cline{5-7} 
\multicolumn{1}{|c|}{p}                    & \multicolumn{1}{c|}{($b_f,b_c$)} & \multicolumn{1}{c|}{$Mo^*$}  & \multicolumn{1}{c|}{} & \multicolumn{1}{c|}{p}                    & \multicolumn{1}{c|}{($b_f,b_c$)} & \multicolumn{1}{c|}{$Mo^*$}  \\ \cline{1-3} \cline{5-7} 
\multicolumn{1}{|c|}{\multirow{2}{*}{0.5}} & \multicolumn{1}{c|}{(30,30)}     & \multicolumn{1}{c|}{2,4,6}   & \multicolumn{1}{c|}{} & \multicolumn{1}{c|}{\multirow{2}{*}{0.5}} & \multicolumn{1}{c|}{(30,30)}     & \multicolumn{1}{c|}{2,4,6}   \\ \cline{2-3} \cline{6-7} 
\multicolumn{1}{|c|}{}                     & \multicolumn{1}{c|}{(40,40)}     & \multicolumn{1}{c|}{1,2,4,6} & \multicolumn{1}{c|}{} & \multicolumn{1}{c|}{}                     & \multicolumn{1}{c|}{(40,40)}     & \multicolumn{1}{c|}{1,2,4,6} \\ \cline{1-3} \cline{5-7} 
\multicolumn{1}{|c|}{\multirow{2}{*}{0.7}} & \multicolumn{1}{c|}{(30,30)}     & \multicolumn{1}{c|}{1,4,6}   & \multicolumn{1}{c|}{} & \multicolumn{1}{c|}{\multirow{2}{*}{0.7}} & \multicolumn{1}{c|}{(30,30)}     & \multicolumn{1}{c|}{1,4,6}   \\ \cline{2-3} \cline{6-7} 
\multicolumn{1}{|c|}{}                     & \multicolumn{1}{c|}{(40,40)}     & \multicolumn{1}{c|}{2,3,4,6} & \multicolumn{1}{c|}{} & \multicolumn{1}{c|}{}                     & \multicolumn{1}{c|}{(40,40)}     & \multicolumn{1}{c|}{2,3,4,6} \\ \cline{1-3} \cline{5-7} 
\multicolumn{1}{|c|}{\multirow{2}{*}{0.9}} & \multicolumn{1}{c|}{(30,30)}     & \multicolumn{1}{c|}{4,5,6}   & \multicolumn{1}{c|}{} & \multicolumn{1}{c|}{\multirow{2}{*}{0.9}} & \multicolumn{1}{c|}{(30,30)}     & \multicolumn{1}{c|}{4,5,6}   \\ \cline{2-3} \cline{6-7} 
\multicolumn{1}{|c|}{}                     & \multicolumn{1}{c|}{(40,40)}     & \multicolumn{1}{c|}{1,4,5,6} & \multicolumn{1}{c|}{} & \multicolumn{1}{c|}{}                     & \multicolumn{1}{c|}{(40,40)}     & \multicolumn{1}{c|}{1,4,5,6} \\ \cline{1-3} \cline{5-7} 
                                           &                                  &                              &                       &                                           &                                  &                              \\
\multicolumn{3}{c}{\textbf{Scenario c)}}                                                                     &                       & \multicolumn{3}{c}{\textbf{Scenario d)}}                                                                    \\ \cline{1-3} \cline{5-7} 
\multicolumn{1}{|c|}{p}                    & \multicolumn{1}{c|}{($b_f,b_c$)} & \multicolumn{1}{c|}{$Mo^*$}  & \multicolumn{1}{c|}{} & \multicolumn{1}{c|}{p}                    & \multicolumn{1}{c|}{($b_f,b_c$)} & \multicolumn{1}{c|}{$Mo^*$}  \\ \cline{1-3} \cline{5-7} 
\multicolumn{1}{|c|}{\multirow{2}{*}{0.5}} & \multicolumn{1}{c|}{(30,30)}     & \multicolumn{1}{c|}{1,2,6}   & \multicolumn{1}{c|}{} & \multicolumn{1}{c|}{\multirow{2}{*}{0.5}} & \multicolumn{1}{c|}{(30,30)}     & \multicolumn{1}{c|}{1,3,6}   \\ \cline{2-3} \cline{6-7} 
\multicolumn{1}{|c|}{}                     & \multicolumn{1}{c|}{(40,40)}     & \multicolumn{1}{c|}{1,3}     & \multicolumn{1}{c|}{} & \multicolumn{1}{c|}{}                     & \multicolumn{1}{c|}{(40,40)}     & \multicolumn{1}{c|}{1,2,3,6} \\ \cline{1-3} \cline{5-7} 
\multicolumn{1}{|c|}{\multirow{2}{*}{0.7}} & \multicolumn{1}{c|}{(30,30)}     & \multicolumn{1}{c|}{1,3,6}   & \multicolumn{1}{c|}{} & \multicolumn{1}{c|}{\multirow{2}{*}{0.7}} & \multicolumn{1}{c|}{(30,30)}     & \multicolumn{1}{c|}{1,2,6}   \\ \cline{2-3} \cline{6-7} 
\multicolumn{1}{|c|}{}                     & \multicolumn{1}{c|}{(40,40)}     & \multicolumn{1}{c|}{1,2,3,6} & \multicolumn{1}{c|}{} & \multicolumn{1}{c|}{}                     & \multicolumn{1}{c|}{(40,40)}     & \multicolumn{1}{c|}{1,2,5,6} \\ \cline{1-3} \cline{5-7} 
\multicolumn{1}{|c|}{\multirow{2}{*}{0.9}} & \multicolumn{1}{c|}{(30,30)}     & \multicolumn{1}{c|}{1,3,6}   & \multicolumn{1}{c|}{} & \multicolumn{1}{c|}{\multirow{2}{*}{0.9}} & \multicolumn{1}{c|}{(30,30)}     & \multicolumn{1}{c|}{1,2,6}   \\ \cline{2-3} \cline{6-7} 
\multicolumn{1}{|c|}{}                     & \multicolumn{1}{c|}{(40,40)}     & \multicolumn{1}{c|}{1,2,3,6} & \multicolumn{1}{c|}{} & \multicolumn{1}{c|}{}                     & \multicolumn{1}{c|}{(40,40)}     & \multicolumn{1}{c|}{1,2,3,6} \\ \cline{1-3} \cline{5-7} 
                                           &                                  &                              &                       &                                           &                                  &                              \\
\multicolumn{3}{c}{\textbf{Scenario e)}}                                                                     &                       & \multicolumn{3}{c}{\textbf{Scenario f)}}                                                                    \\ \cline{1-3} \cline{5-7} 
\multicolumn{1}{|c|}{p}                    & \multicolumn{1}{c|}{($b_f,b_c$)} & \multicolumn{1}{c|}{$Mo^*$}  & \multicolumn{1}{c|}{} & \multicolumn{1}{c|}{p}                    & \multicolumn{1}{c|}{($b_f,b_c$)} & \multicolumn{1}{c|}{$Mo^*$}  \\ \cline{1-3} \cline{5-7} 
\multicolumn{1}{|c|}{\multirow{2}{*}{0.5}} & \multicolumn{1}{c|}{(30,30)}     & \multicolumn{1}{c|}{2,3,6}   & \multicolumn{1}{c|}{} & \multicolumn{1}{c|}{\multirow{2}{*}{0.5}} & \multicolumn{1}{c|}{(30,30)}     & \multicolumn{1}{c|}{1,2,6}   \\ \cline{2-3} \cline{6-7} 
\multicolumn{1}{|c|}{}                     & \multicolumn{1}{c|}{(40,40)}     & \multicolumn{1}{c|}{1,2,3,6} & \multicolumn{1}{c|}{} & \multicolumn{1}{c|}{}                     & \multicolumn{1}{c|}{(40,40)}     & \multicolumn{1}{c|}{1,2,3,6} \\ \cline{1-3} \cline{5-7} 
\multicolumn{1}{|c|}{\multirow{2}{*}{0.7}} & \multicolumn{1}{c|}{(30,30)}     & \multicolumn{1}{c|}{2,3,6}   & \multicolumn{1}{c|}{} & \multicolumn{1}{c|}{\multirow{2}{*}{0.7}} & \multicolumn{1}{c|}{(30,30)}     & \multicolumn{1}{c|}{1,2,6}   \\ \cline{2-3} \cline{6-7} 
\multicolumn{1}{|c|}{}                     & \multicolumn{1}{c|}{(40,40)}     & \multicolumn{1}{c|}{2,6}     & \multicolumn{1}{c|}{} & \multicolumn{1}{c|}{}                     & \multicolumn{1}{c|}{(40,40)}     & \multicolumn{1}{c|}{1,2,3,6} \\ \cline{1-3} \cline{5-7} 
\multicolumn{1}{|c|}{\multirow{2}{*}{0.9}} & \multicolumn{1}{c|}{(30,30)}     & \multicolumn{1}{c|}{2,3,6}   & \multicolumn{1}{c|}{} & \multicolumn{1}{c|}{\multirow{2}{*}{0.9}} & \multicolumn{1}{c|}{(30,30)}     & \multicolumn{1}{c|}{1,2,6}   \\ \cline{2-3} \cline{6-7} 
\multicolumn{1}{|c|}{}                     & \multicolumn{1}{c|}{(40,40)}     & \multicolumn{1}{c|}{2,3,5,6} & \multicolumn{1}{c|}{} & \multicolumn{1}{c|}{}                     & \multicolumn{1}{c|}{(40,40)}     & \multicolumn{1}{c|}{1,2,4,6} \\ \cline{1-3} \cline{5-7} 
\end{tabular}%
}
\caption{Optimal monitoring sets of the Use Cases}
\label{tab:mdp-usecase-optiset}
\end{table}

\textbf{Scenarios a) and b)\\}
We note that nodes 4 and 6 are always chosen in the monitoring.
Node 1 also often appears as a good candidate for a fourth node if it is not already chosen third.
With $p=0.7$ we observe that third and fourth nodes do not coincide between (30, 30) and (40, 40) budgets.
This suggests that combining nodes 2 and 3 increases their individual performance, (\ie nodes 2 and 3 perform better than nodes 1 and 2 or 1 and 3).
We suppose this is because node 1 is surrounded by nodes 2 and 3 in the topology.
% This is explained because other budgets impact the individual importance of each node.

\textbf{Scenarios c) and d)\\}
One notable result is in scenario c) for $p=0.5,  (b_f,b_c) = (40,40)$ where the optimal monitoring set is \{1, 3\} which could have two extra nodes ensuring the monitoring. However, it is also worth noting that for $(b_f,b_c)=(40,40)$, the optimal set does not include node 6. The explanation comes from the fact that it takes less time to establish the path between the victim's network and the extraction point because of the full-meshed topology. The reward is based on the nodes of the attack path when the attack is completed. Therefore attacking extra nodes does not impact the reward. We conclude that the low detection probability ($p=0.5$) coupled to a full-meshed topology may not be the ideal conditions to improve the security of the migration process.


\textbf{Scenarios e) and f)}\\
There is one  notable result, in scenario e) for $p=0.7,  (b_f,b_c) = (40,40)$ where the optimal monitoring set is \{2, 6\} which could also have two extra nodes supporting the monitoring. In opposition to the counter-intuitive result of scenario c), we are here considering a higher detection percentage. Moreover, the nodes considered in this solution set are the nodes that see most of the attacks, one being the source of all attacks, and the other being part of the migrated virtual network and communicating with almost all the other nodes in the infrastructure. The analysis of the optimal policy of this use case shows that part of the counter intuitive result is caused by removing either node 3 or 4 from the monitoring, which is understandable because it would not contribute to the detection of the remaining attacks. Removing monitoring becomes equivalent to the action of doing nothing, with the unintended and unexpected upside of reducing the performance impact on the infrastructure.
The other part of the counter intuitive result (not shown in Table~\ref{tab:mdp-usecase-optiset}) is due to removing node 6, the source of attacks from the monitoring. It is worth noting that this particular case represents a negligible portion of the solution space, and most of the counter intuitive results are due to the previous explanation. Considering that node 6 is removed at a point where the consequences of attacks will be minimal, removing the monitoring at this point was deemed equivalent to the ``do nothing" action by the MDP.


