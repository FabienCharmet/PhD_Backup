When solving a problem using an MDP, the solution is a dynamic proposition to choose actions as the system evolves.
However, from a technical aspect, the defender needs to have the nodes already monitoring the infrastructure before starting the migration process.
It becomes necessary to translate the dynamic answer of the MDP into a static \textit{a priori} deployment.
After determining the individual importance of each node, we propose to determine the optimal set of monitoring nodes.

The main difference is that each node was evaluated based on all the possible budget combinations, whereas what is defined here is a particular answer for a specific budget.
For each budget, the maximum reward is $\frac{b_f}{c_f} \sum\limits_{i \in \textbf{N}}V_i(1-\alpha) $ which corresponds to the corner case where the attacker never launched an attack, and we can evaluate the efficiency of the  monitoring nodes thanks to the associated reward.
Even if a particular monitoring set achieves close to the maximum reward, it is also because this monitoring set is the optimal solution to a specific subset of all possible attacks.
We propose to determine the optimal monitoring state for each budget by weighting the reward they achieve with their occupation of the total solution space.

% We note $S_{\text{abs}}$ the set of absorbing states, 
We note $S^{\text{Mo}}_{\text{abs}}$ the set of absorbing states with a common  monitoring set $Mo$, $\rho(Mo)$ the percentage of presence of set $Mo$ in the solution space and $R(Mo)$ the reward of monitoring set $Mo$.
% \begin{equation}
%     R(Mo) = \sum\limits_{s \in S_{Mo}^{abs}}\rho(Mo^s)R(Mo^s)
% \end{equation}
Then we propose to choose the optimal monitoring set $Mo^*$ with:
\begin{equation}
    Mo^* = \argmax\limits_{\text{Mo}} \left \{\sum\limits_{s \in S^{\text{Mo}}_{\text{abs}}}\rho(Mo^s)R(Mo^s) \right \}
\end{equation}

The optimal monitoring set is determined by choosing the set giving the maximum weighted reward.

% \CK{Les tables sont un peu enormes, reduis un peu leur taille}

% Please add the following required packages to your document preamble:
% \usepackage{multirow}
% \usepackage{graphicx}
\begin{table}[h]
\resizebox{\textwidth}{!}{%
\begin{tabular}{ccccccc}
\multicolumn{3}{c}{\textbf{Scenario a)}}                                                                     &                       & \multicolumn{3}{c}{\textbf{Scenario b)}}                                                                    \\ \cline{1-3} \cline{5-7} 
\multicolumn{1}{|c|}{p}                    & \multicolumn{1}{c|}{($b_f,b_c$)} & \multicolumn{1}{c|}{$Mo^*$}  & \multicolumn{1}{c|}{} & \multicolumn{1}{c|}{p}                    & \multicolumn{1}{c|}{($b_f,b_c$)} & \multicolumn{1}{c|}{$Mo^*$}  \\ \cline{1-3} \cline{5-7} 
\multicolumn{1}{|c|}{\multirow{2}{*}{0.5}} & \multicolumn{1}{c|}{(30,30)}     & \multicolumn{1}{c|}{2,4,6}   & \multicolumn{1}{c|}{} & \multicolumn{1}{c|}{\multirow{2}{*}{0.5}} & \multicolumn{1}{c|}{(30,30)}     & \multicolumn{1}{c|}{2,4,6}   \\ \cline{2-3} \cline{6-7} 
\multicolumn{1}{|c|}{}                     & \multicolumn{1}{c|}{(40,40)}     & \multicolumn{1}{c|}{1,2,4,6} & \multicolumn{1}{c|}{} & \multicolumn{1}{c|}{}                     & \multicolumn{1}{c|}{(40,40)}     & \multicolumn{1}{c|}{1,2,4,6} \\ \cline{1-3} \cline{5-7} 
\multicolumn{1}{|c|}{\multirow{2}{*}{0.7}} & \multicolumn{1}{c|}{(30,30)}     & \multicolumn{1}{c|}{1,4,6}   & \multicolumn{1}{c|}{} & \multicolumn{1}{c|}{\multirow{2}{*}{0.7}} & \multicolumn{1}{c|}{(30,30)}     & \multicolumn{1}{c|}{1,4,6}   \\ \cline{2-3} \cline{6-7} 
\multicolumn{1}{|c|}{}                     & \multicolumn{1}{c|}{(40,40)}     & \multicolumn{1}{c|}{2,3,4,6} & \multicolumn{1}{c|}{} & \multicolumn{1}{c|}{}                     & \multicolumn{1}{c|}{(40,40)}     & \multicolumn{1}{c|}{2,3,4,6} \\ \cline{1-3} \cline{5-7} 
\multicolumn{1}{|c|}{\multirow{2}{*}{0.9}} & \multicolumn{1}{c|}{(30,30)}     & \multicolumn{1}{c|}{4,5,6}   & \multicolumn{1}{c|}{} & \multicolumn{1}{c|}{\multirow{2}{*}{0.9}} & \multicolumn{1}{c|}{(30,30)}     & \multicolumn{1}{c|}{4,5,6}   \\ \cline{2-3} \cline{6-7} 
\multicolumn{1}{|c|}{}                     & \multicolumn{1}{c|}{(40,40)}     & \multicolumn{1}{c|}{1,4,5,6} & \multicolumn{1}{c|}{} & \multicolumn{1}{c|}{}                     & \multicolumn{1}{c|}{(40,40)}     & \multicolumn{1}{c|}{1,4,5,6} \\ \cline{1-3} \cline{5-7} 
                                           &                                  &                              &                       &                                           &                                  &                              \\
\multicolumn{3}{c}{\textbf{Scenario c)}}                                                                     &                       & \multicolumn{3}{c}{\textbf{Scenario d)}}                                                                    \\ \cline{1-3} \cline{5-7} 
\multicolumn{1}{|c|}{p}                    & \multicolumn{1}{c|}{($b_f,b_c$)} & \multicolumn{1}{c|}{$Mo^*$}  & \multicolumn{1}{c|}{} & \multicolumn{1}{c|}{p}                    & \multicolumn{1}{c|}{($b_f,b_c$)} & \multicolumn{1}{c|}{$Mo^*$}  \\ \cline{1-3} \cline{5-7} 
\multicolumn{1}{|c|}{\multirow{2}{*}{0.5}} & \multicolumn{1}{c|}{(30,30)}     & \multicolumn{1}{c|}{1,2,6}   & \multicolumn{1}{c|}{} & \multicolumn{1}{c|}{\multirow{2}{*}{0.5}} & \multicolumn{1}{c|}{(30,30)}     & \multicolumn{1}{c|}{1,3,6}   \\ \cline{2-3} \cline{6-7} 
\multicolumn{1}{|c|}{}                     & \multicolumn{1}{c|}{(40,40)}     & \multicolumn{1}{c|}{1,3}     & \multicolumn{1}{c|}{} & \multicolumn{1}{c|}{}                     & \multicolumn{1}{c|}{(40,40)}     & \multicolumn{1}{c|}{1,2,3,6} \\ \cline{1-3} \cline{5-7} 
\multicolumn{1}{|c|}{\multirow{2}{*}{0.7}} & \multicolumn{1}{c|}{(30,30)}     & \multicolumn{1}{c|}{1,3,6}   & \multicolumn{1}{c|}{} & \multicolumn{1}{c|}{\multirow{2}{*}{0.7}} & \multicolumn{1}{c|}{(30,30)}     & \multicolumn{1}{c|}{1,2,6}   \\ \cline{2-3} \cline{6-7} 
\multicolumn{1}{|c|}{}                     & \multicolumn{1}{c|}{(40,40)}     & \multicolumn{1}{c|}{1,2,3,6} & \multicolumn{1}{c|}{} & \multicolumn{1}{c|}{}                     & \multicolumn{1}{c|}{(40,40)}     & \multicolumn{1}{c|}{1,2,5,6} \\ \cline{1-3} \cline{5-7} 
\multicolumn{1}{|c|}{\multirow{2}{*}{0.9}} & \multicolumn{1}{c|}{(30,30)}     & \multicolumn{1}{c|}{1,3,6}   & \multicolumn{1}{c|}{} & \multicolumn{1}{c|}{\multirow{2}{*}{0.9}} & \multicolumn{1}{c|}{(30,30)}     & \multicolumn{1}{c|}{1,2,6}   \\ \cline{2-3} \cline{6-7} 
\multicolumn{1}{|c|}{}                     & \multicolumn{1}{c|}{(40,40)}     & \multicolumn{1}{c|}{1,2,3,6} & \multicolumn{1}{c|}{} & \multicolumn{1}{c|}{}                     & \multicolumn{1}{c|}{(40,40)}     & \multicolumn{1}{c|}{1,2,3,6} \\ \cline{1-3} \cline{5-7} 
                                           &                                  &                              &                       &                                           &                                  &                              \\
\multicolumn{3}{c}{\textbf{Scenario e)}}                                                                     &                       & \multicolumn{3}{c}{\textbf{Scenario f)}}                                                                    \\ \cline{1-3} \cline{5-7} 
\multicolumn{1}{|c|}{p}                    & \multicolumn{1}{c|}{($b_f,b_c$)} & \multicolumn{1}{c|}{$Mo^*$}  & \multicolumn{1}{c|}{} & \multicolumn{1}{c|}{p}                    & \multicolumn{1}{c|}{($b_f,b_c$)} & \multicolumn{1}{c|}{$Mo^*$}  \\ \cline{1-3} \cline{5-7} 
\multicolumn{1}{|c|}{\multirow{2}{*}{0.5}} & \multicolumn{1}{c|}{(30,30)}     & \multicolumn{1}{c|}{2,3,6}   & \multicolumn{1}{c|}{} & \multicolumn{1}{c|}{\multirow{2}{*}{0.5}} & \multicolumn{1}{c|}{(30,30)}     & \multicolumn{1}{c|}{1,2,6}   \\ \cline{2-3} \cline{6-7} 
\multicolumn{1}{|c|}{}                     & \multicolumn{1}{c|}{(40,40)}     & \multicolumn{1}{c|}{1,2,3,6} & \multicolumn{1}{c|}{} & \multicolumn{1}{c|}{}                     & \multicolumn{1}{c|}{(40,40)}     & \multicolumn{1}{c|}{1,2,3,6} \\ \cline{1-3} \cline{5-7} 
\multicolumn{1}{|c|}{\multirow{2}{*}{0.7}} & \multicolumn{1}{c|}{(30,30)}     & \multicolumn{1}{c|}{2,3,6}   & \multicolumn{1}{c|}{} & \multicolumn{1}{c|}{\multirow{2}{*}{0.7}} & \multicolumn{1}{c|}{(30,30)}     & \multicolumn{1}{c|}{1,2,6}   \\ \cline{2-3} \cline{6-7} 
\multicolumn{1}{|c|}{}                     & \multicolumn{1}{c|}{(40,40)}     & \multicolumn{1}{c|}{2,6}     & \multicolumn{1}{c|}{} & \multicolumn{1}{c|}{}                     & \multicolumn{1}{c|}{(40,40)}     & \multicolumn{1}{c|}{1,2,3,6} \\ \cline{1-3} \cline{5-7} 
\multicolumn{1}{|c|}{\multirow{2}{*}{0.9}} & \multicolumn{1}{c|}{(30,30)}     & \multicolumn{1}{c|}{2,3,6}   & \multicolumn{1}{c|}{} & \multicolumn{1}{c|}{\multirow{2}{*}{0.9}} & \multicolumn{1}{c|}{(30,30)}     & \multicolumn{1}{c|}{1,2,6}   \\ \cline{2-3} \cline{6-7} 
\multicolumn{1}{|c|}{}                     & \multicolumn{1}{c|}{(40,40)}     & \multicolumn{1}{c|}{2,3,5,6} & \multicolumn{1}{c|}{} & \multicolumn{1}{c|}{}                     & \multicolumn{1}{c|}{(40,40)}     & \multicolumn{1}{c|}{1,2,4,6} \\ \cline{1-3} \cline{5-7} 
\end{tabular}%
}
\caption{Optimal monitoring sets of the use cases}
\label{tab:mdp-usecase-optiset}
\end{table}
% \FC{a relire}

\subsection{Experimental results: Monitoring sets}

We describe the optimal monitoring sets for each use case.
Results for all scenarios are summarized in Table~\ref{tab:mdp-usecase-optiset}. 

\subsubsection*{Balanced topology: Scenarios a) and b)}
We note that node 6 is always chosen in the monitoring, as it is the main source of attacks.
Node 4 also always appears as a good candidate for a fourth node if it is not already chosen third.
In both scenarios, we observe that third and fourth nodes rarely coincide between (30, 30) and (40, 40) budgets.
In scenario a), for a (40,40) budget at $p=0.7$,  nodes 2 and 3 are selected instead of reusing the node 1 that was selected for the (30,30) budget.
This suggests that combining two nodes increases their individual performance.
In scenario b) at $p=0.5$, we obtain a similar result where nodes 3 and 4 are considered over node 2.

% We suppose this is because node 1 is surrounded by nodes 2 and 3 in the topology.
% This is explained because other budgets impact the individual importance of each node.

\subsubsection*{Full-mesh topology: Scenarios c) and d)}
One notable result is in scenario c) for $p=0.5,  (b_f,b_c) = (40,40)$ and in scenario d) for $p=0.7,  (b_f,b_c) = (40,40)$, where the optimal monitoring set is \{1,2,5,6\} which does not include node 3 like other budgets.  The explanation comes from the fact that it takes less time to establish the path between the victim's network and the extraction point because of the full-mesh topology. There are only two nodes capable of detecting an attack every time it is started, thus the monitoring of extra nodes does not contribute to a better reward.

We conclude that our results confirm the intuition that a low detection probability ($p=0.5$) coupled to a full-mesh topology may not be the ideal conditions to improve the security of the migration process. 


\subsubsection*{Central topology: Scenarios e) and f)}
Both scenarios exhibit similar results, where nodes 2 and 6 are consistently picked every time.
Indeed, node 6 is the source of attacks, and the MDP also selects node 2 because it is on the path of most attacks as a valuable node.
In scenario e) node 3 is often picked first over node 1, in opposition to scenario f).
This choice is due to the second attack source becoming more valuable in scenario f).

\subsection{Summary}
We have confirmed in every scenario the intuition that the attack source must be part of the monitoring set. Results also show that the second attacker should be monitored.
In some cases, it seems that combining the monitoring of two nodes increases their individual performance (\eg Scenario a) at $p=0.9$, node 5 for budget (30,30) and nodes 2 and 3 for budget (40,40) ).
% Overall, the model outlines which nodes take part in m
% \FC{Verifier p=0.7 scenario e)}

% The other part of the counter intuitive result (not shown in Table~\ref{tab:mdp-usecase-optiset}) is due to removing node 6, the source of attacks from the monitoring. It is worth noting that this particular case represents a negligible portion of the solution space, and most of the counter intuitive results are due to the previous explanation. Considering that node 6 is removed at a point where the consequences of attacks will be minimal, it was deemed equivalent to the ``do nothing" action by the MDP. \CK{Je trouve ce paragraphe (a partir de "The other part of...") pas clair du tout et je ne pense pas qu'il apporte grand chose, je suggere de le supprimer}


