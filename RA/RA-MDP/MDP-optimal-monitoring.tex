When solving a problem using an MDP, the solution is a dynamic proposition to choose actions as the system evolves.
However, from a technical aspect, the defender needs to have the nodes already monitoring the infrastructure before starting the migration process.
It becomes necessary to translate the dynamic answer of the MDP into a static \textit{a priori} deployment.
After determining the individual importance of each node, we propose to determine the optimal set of monitoring nodes.

The main difference is that each node was evaluated based on all the possible budget combinations, whereas what is defined here is a particular answer for a specific budget.
For each budget, the maximum reward is $\frac{b_f}{c_a} \sum\limits_{i \in \textbf{N}}V_i $ which corresponds to the corner case where the attacker never launched an attack, and we can evaluate the efficiency of the   monitoring nodes thanks to the associated reward.
Even if a particular monitoring set achieves close to the maximum reward, it is also because the set is tailored to a subset of all possible attacks.
We propose to determine the optimal monitoring state for each budget by weighting the reward they achieve with their occupation of the total solution space.

We note $S_{\text{abs}}$ the set of absorbing states, $S^{\text{Mo}}_{\text{abs}}$ the set of absorbing states with a common  monitoring set $Mo$, $\rho(Mo)$ the percentage of presence of set $Mo$ in the solution space and $R(Mo)$ the reward of monitoring set $Mo$.
% \begin{equation}
%     R(Mo) = \sum\limits_{s \in S_{Mo}^{abs}}\rho(Mo^s)R(Mo^s)
% \end{equation}
Then we propose to choose the optimal monitoring set $Mo^*$ with:
\begin{equation}
    Mo^* = \argmax\limits_{\text{Mo} \in S_{\text{abs}}} \left \{\sum\limits_{s \in S^{\text{Mo}}_{\text{abs}}}\rho(Mo^s)R(Mo^s) \right \}
\end{equation}

We present the results for scenario a) in Table~\ref{tab:mdp-usecase-optiset}0.
We observe that nodes 4 and 6 are always chosen in the monitoring, which corresponds to Fig.~\ref{fig:nodeimp_single}.
Node 1 also often appears as a good candidate for a fourth node if it is not already chosen third.
With $p=0.7$ we observe that third and fourth nodes do not coincidate between (30,30) and (40,40) budgets.
This suggests that the combining two nodes increases their individual performance.
(Node 1 is surrounded by node 2 and 3 in the topology).
% This is explained because other budgets impact the individual importance of each node.

% Please add the following required packages to your document preamble:
% \usepackage{multirow}
% \usepackage{graphicx}
\begin{table}[h]
\resizebox{\textwidth}{!}{%
\begin{tabular}{ccccccc}
\multicolumn{3}{c}{\textbf{Scenario a)}}                                                                     &                       & \multicolumn{3}{c}{\textbf{Scenario b)}}                                                                    \\ \cline{1-3} \cline{5-7} 
\multicolumn{1}{|c|}{p}                    & \multicolumn{1}{c|}{($b_f,b_c$)} & \multicolumn{1}{c|}{$Mo^*$}  & \multicolumn{1}{c|}{} & \multicolumn{1}{c|}{p}                    & \multicolumn{1}{c|}{($b_f,b_c$)} & \multicolumn{1}{c|}{$Mo^*$}  \\ \cline{1-3} \cline{5-7} 
\multicolumn{1}{|c|}{\multirow{2}{*}{0.5}} & \multicolumn{1}{c|}{(30,30)}     & \multicolumn{1}{c|}{2,4,6}   & \multicolumn{1}{c|}{} & \multicolumn{1}{c|}{\multirow{2}{*}{0.5}} & \multicolumn{1}{c|}{(30,30)}     & \multicolumn{1}{c|}{2,4,6}   \\ \cline{2-3} \cline{6-7} 
\multicolumn{1}{|c|}{}                     & \multicolumn{1}{c|}{(40,40)}     & \multicolumn{1}{c|}{1,2,4,6} & \multicolumn{1}{c|}{} & \multicolumn{1}{c|}{}                     & \multicolumn{1}{c|}{(40,40)}     & \multicolumn{1}{c|}{1,2,4,6} \\ \cline{1-3} \cline{5-7} 
\multicolumn{1}{|c|}{\multirow{2}{*}{0.7}} & \multicolumn{1}{c|}{(30,30)}     & \multicolumn{1}{c|}{1,4,6}   & \multicolumn{1}{c|}{} & \multicolumn{1}{c|}{\multirow{2}{*}{0.7}} & \multicolumn{1}{c|}{(30,30)}     & \multicolumn{1}{c|}{1,4,6}   \\ \cline{2-3} \cline{6-7} 
\multicolumn{1}{|c|}{}                     & \multicolumn{1}{c|}{(40,40)}     & \multicolumn{1}{c|}{2,3,4,6} & \multicolumn{1}{c|}{} & \multicolumn{1}{c|}{}                     & \multicolumn{1}{c|}{(40,40)}     & \multicolumn{1}{c|}{2,3,4,6} \\ \cline{1-3} \cline{5-7} 
\multicolumn{1}{|c|}{\multirow{2}{*}{0.9}} & \multicolumn{1}{c|}{(30,30)}     & \multicolumn{1}{c|}{4,5,6}   & \multicolumn{1}{c|}{} & \multicolumn{1}{c|}{\multirow{2}{*}{0.9}} & \multicolumn{1}{c|}{(30,30)}     & \multicolumn{1}{c|}{4,5,6}   \\ \cline{2-3} \cline{6-7} 
\multicolumn{1}{|c|}{}                     & \multicolumn{1}{c|}{(40,40)}     & \multicolumn{1}{c|}{1,4,5,6} & \multicolumn{1}{c|}{} & \multicolumn{1}{c|}{}                     & \multicolumn{1}{c|}{(40,40)}     & \multicolumn{1}{c|}{1,4,5,6} \\ \cline{1-3} \cline{5-7} 
                                           &                                  &                              &                       &                                           &                                  &                              \\
\multicolumn{3}{c}{\textbf{Scenario c)}}                                                                     &                       & \multicolumn{3}{c}{\textbf{Scenario d)}}                                                                    \\ \cline{1-3} \cline{5-7} 
\multicolumn{1}{|c|}{p}                    & \multicolumn{1}{c|}{($b_f,b_c$)} & \multicolumn{1}{c|}{$Mo^*$}  & \multicolumn{1}{c|}{} & \multicolumn{1}{c|}{p}                    & \multicolumn{1}{c|}{($b_f,b_c$)} & \multicolumn{1}{c|}{$Mo^*$}  \\ \cline{1-3} \cline{5-7} 
\multicolumn{1}{|c|}{\multirow{2}{*}{0.5}} & \multicolumn{1}{c|}{(30,30)}     & \multicolumn{1}{c|}{1,2,6}   & \multicolumn{1}{c|}{} & \multicolumn{1}{c|}{\multirow{2}{*}{0.5}} & \multicolumn{1}{c|}{(30,30)}     & \multicolumn{1}{c|}{1,3,6}   \\ \cline{2-3} \cline{6-7} 
\multicolumn{1}{|c|}{}                     & \multicolumn{1}{c|}{(40,40)}     & \multicolumn{1}{c|}{1,3}     & \multicolumn{1}{c|}{} & \multicolumn{1}{c|}{}                     & \multicolumn{1}{c|}{(40,40)}     & \multicolumn{1}{c|}{1,2,3,6} \\ \cline{1-3} \cline{5-7} 
\multicolumn{1}{|c|}{\multirow{2}{*}{0.7}} & \multicolumn{1}{c|}{(30,30)}     & \multicolumn{1}{c|}{1,3,6}   & \multicolumn{1}{c|}{} & \multicolumn{1}{c|}{\multirow{2}{*}{0.7}} & \multicolumn{1}{c|}{(30,30)}     & \multicolumn{1}{c|}{1,2,6}   \\ \cline{2-3} \cline{6-7} 
\multicolumn{1}{|c|}{}                     & \multicolumn{1}{c|}{(40,40)}     & \multicolumn{1}{c|}{1,2,3,6} & \multicolumn{1}{c|}{} & \multicolumn{1}{c|}{}                     & \multicolumn{1}{c|}{(40,40)}     & \multicolumn{1}{c|}{1,2,5,6} \\ \cline{1-3} \cline{5-7} 
\multicolumn{1}{|c|}{\multirow{2}{*}{0.9}} & \multicolumn{1}{c|}{(30,30)}     & \multicolumn{1}{c|}{1,3,6}   & \multicolumn{1}{c|}{} & \multicolumn{1}{c|}{\multirow{2}{*}{0.9}} & \multicolumn{1}{c|}{(30,30)}     & \multicolumn{1}{c|}{1,2,6}   \\ \cline{2-3} \cline{6-7} 
\multicolumn{1}{|c|}{}                     & \multicolumn{1}{c|}{(40,40)}     & \multicolumn{1}{c|}{1,2,3,6} & \multicolumn{1}{c|}{} & \multicolumn{1}{c|}{}                     & \multicolumn{1}{c|}{(40,40)}     & \multicolumn{1}{c|}{1,2,3,6} \\ \cline{1-3} \cline{5-7} 
                                           &                                  &                              &                       &                                           &                                  &                              \\
\multicolumn{3}{c}{\textbf{Scenario e)}}                                                                     &                       & \multicolumn{3}{c}{\textbf{Scenario f)}}                                                                    \\ \cline{1-3} \cline{5-7} 
\multicolumn{1}{|c|}{p}                    & \multicolumn{1}{c|}{($b_f,b_c$)} & \multicolumn{1}{c|}{$Mo^*$}  & \multicolumn{1}{c|}{} & \multicolumn{1}{c|}{p}                    & \multicolumn{1}{c|}{($b_f,b_c$)} & \multicolumn{1}{c|}{$Mo^*$}  \\ \cline{1-3} \cline{5-7} 
\multicolumn{1}{|c|}{\multirow{2}{*}{0.5}} & \multicolumn{1}{c|}{(30,30)}     & \multicolumn{1}{c|}{2,3,6}   & \multicolumn{1}{c|}{} & \multicolumn{1}{c|}{\multirow{2}{*}{0.5}} & \multicolumn{1}{c|}{(30,30)}     & \multicolumn{1}{c|}{1,2,6}   \\ \cline{2-3} \cline{6-7} 
\multicolumn{1}{|c|}{}                     & \multicolumn{1}{c|}{(40,40)}     & \multicolumn{1}{c|}{1,2,3,6} & \multicolumn{1}{c|}{} & \multicolumn{1}{c|}{}                     & \multicolumn{1}{c|}{(40,40)}     & \multicolumn{1}{c|}{1,2,3,6} \\ \cline{1-3} \cline{5-7} 
\multicolumn{1}{|c|}{\multirow{2}{*}{0.7}} & \multicolumn{1}{c|}{(30,30)}     & \multicolumn{1}{c|}{2,3,6}   & \multicolumn{1}{c|}{} & \multicolumn{1}{c|}{\multirow{2}{*}{0.7}} & \multicolumn{1}{c|}{(30,30)}     & \multicolumn{1}{c|}{1,2,6}   \\ \cline{2-3} \cline{6-7} 
\multicolumn{1}{|c|}{}                     & \multicolumn{1}{c|}{(40,40)}     & \multicolumn{1}{c|}{2,6}     & \multicolumn{1}{c|}{} & \multicolumn{1}{c|}{}                     & \multicolumn{1}{c|}{(40,40)}     & \multicolumn{1}{c|}{1,2,3,6} \\ \cline{1-3} \cline{5-7} 
\multicolumn{1}{|c|}{\multirow{2}{*}{0.9}} & \multicolumn{1}{c|}{(30,30)}     & \multicolumn{1}{c|}{2,3,6}   & \multicolumn{1}{c|}{} & \multicolumn{1}{c|}{\multirow{2}{*}{0.9}} & \multicolumn{1}{c|}{(30,30)}     & \multicolumn{1}{c|}{1,2,6}   \\ \cline{2-3} \cline{6-7} 
\multicolumn{1}{|c|}{}                     & \multicolumn{1}{c|}{(40,40)}     & \multicolumn{1}{c|}{2,3,5,6} & \multicolumn{1}{c|}{} & \multicolumn{1}{c|}{}                     & \multicolumn{1}{c|}{(40,40)}     & \multicolumn{1}{c|}{1,2,4,6} \\ \cline{1-3} \cline{5-7} 
\end{tabular}%
}
\caption{Optimal monitoring sets of the use cases}
\label{tab:mdp-usecase-optiset}
\end{table}