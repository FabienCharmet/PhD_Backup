\label{sec:mdp-system-hypotheses}
We describe here the assumptions made about the infrastructure and the migration process.

% \textbf{Migration}
\begin{itemize}
    \item
    The migration will deploy in the target physical substrate all the flow rules necessary to operate the Virtual Network properly. In this chapter the expression ``migrating a node" means deploying configuration rules in a physical node.
    
    \item
    We consider that the migration of the virtual nodes is sequential~\cite{Lime-Ghorbani2014}, thus the nodes will be migrated one at a time.
    We suppose that both Virtual Network and physical infrastructure are static (\ie the topology does not change over time).
    
    \item All nodes in the original substrate have been fully compromised by the attacker and thus are not considered as candidates for the destination substrate.  
    This assumption is reinforced by the fact that forcing all the resources to be reallocated could be leveraged by the attacker in an attempt to have more nodes of the target Virtual Network relocated closer to his Virtual Machines (\ie co-residency).
    Virtual Machines are already subject to such co-residency attacks, as presented by Atya \etal in~\cite{stalling-atya2017,malicious-atya2017}.

    \item
    We also assume that the migration time is uniform across all nodes (\ie no node takes longer to migrate in the infrastructure than another).
    
    \item
     We consider the Virtual Network equivalent to the physical nodes composing its embedding.
     This assumption is used to represent how nodes will be migrated and on which physical node flow rules will be installed. 
\end{itemize}

