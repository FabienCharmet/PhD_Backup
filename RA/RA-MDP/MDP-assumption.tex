\label{sec:mdp-system-hypotheses}
We describe here the assumptions made about the infrastructure and the migration process.

% \textbf{Migration}
\begin{itemize}
    \item
    The migration will deploy in the target physical substrate all the flow rules necessary to operate the Virtual Network properly. We consider that the migration of the virtual nodes is sequential~\cite{Lime-Ghorbani2014}, thus the nodes will be migrated one at a time.
    We suppose that both Virtual Network and physical infrastructure are static (\ie the topology does not change over time).
    
    \item All nodes in the original substrate have been fully compromised by the attacker and thus are not considered as candidates for the destination substrate.  
    This assumption is reinforced by the fact that forcing all the resources to be reallocated could be leveraged by the attacker in an attempt to have the target Virtual Network relocated closer to his virtual machines.
    Virtual machines are already subject to such attacks, as presented by Atya \etal in~\cite{stalling-atya2017,malicious-atya2017}.

    \item
    We also assume that migration time is uniform across all nodes (\ie no node is longer to migrate in the infrastructure than another).
    
    \item
     We simplify the view of the Virtual Network by considering the physical nodes composing the embedding of the VN.
\end{itemize}

