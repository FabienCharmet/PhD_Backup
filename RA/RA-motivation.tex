In the previous chapter we have presented a formal model describe several security aspects of an SDN virtualization infrastructure, and more precisely we considered the virtual network migration process as a specific context in the virtualization.

We made several assumptions for this work, and one of the strongest is the perfect knowledge of the events in the network infrastructure. The proof given by the formal model relies on the fact that we have a complete knowledge of all the events that happened during the migration. As we have explained why during the experimental evaluation of our formal model that assumption holds. However, this assumption is too strong to be considered in a real life environment. Current IDS solutions cannot ensure that all attacks and events impacting the migration can be detected, therefore the perfect knowledge assumption does not hold.

We propose to alleviate this hypothesis for a real life use case with an optimization model for the deployment of the monitoring in the infrastructure.
We consider a virtualization infrastructure, and want to improve the security of the migration process.
Monitoring solutions incur a financial cost for the service provider, as well as a performance impact because computing resources are not dedicated to the virtualization service.
The main objective is to determine which nodes in the infrastructure should support the monitoring function.
The optimal deployment should consider the attacks launched on the migration.
The problem considered here is a typical instance of the resource allocation problem.

Several solution for the resource allocation are presented in the state of the art. 
Linear Programming models the system with a set of equations, and maximize a reward function under a set of constraints.
The main drawback in solving security resource allocation problems with Linear Programming is the lack of consideration for the attacker's capacities in the solution.

The Game Theory formalism is at the other end of the spectrum, where the behavior of both the infrastructure owner and the attacker can be modeled.
The formalism provides different types of games, each corresponding to particular situations and how players choose their action.
Game Theory does not explore the resource allocation problem in an SDN environment, and we expect some applicability in our study.

Markov Decision Processes sit between linear programming and Game Theory, where the model only considers the perspective of one entity, but gives the entity actions to make to interact with the system. The resource allocation problem is not well studied using MDP, but it provides enough granularity in the actions and the transition model to be worth investigating.


