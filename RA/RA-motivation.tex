In the previous chapter, we have presented a theoretical model describing the security of the migration of virtual SDN networks. We focused on describing how Virtual Networks are mapped with the physical infrastructure, and we described how the migration of Virtual Networks is performed. We then described the security aspects of the migration process and how to detect attacks against it. We outlined how an attacker may compromise the migration process in order to gain unlawful access to confidential information or to alter the configuration of migrated Virtual Networks.

We made several assumptions to constrain the scope of this first contribution, and one of the strongest is the omniscience of the detection mechanisms set up in the infrastructure to generate the execution trace. The proof given by the formal model relies on the fact that we have a complete knowledge of all the events that happened during the migration. This assumption holds during the experimental evaluation of our formal model. 
However, this perfect knowledge assumption is too strong to be considered in a real life environment.
Current IDS solutions struggle to process large amounts of information, for two main reasons. The first reason is that detection rules are imperfect and will not detect every intrusion in a voluminous traffic. The second reason is the complexity for an operator to process all the alerts raised by the IDS. This complexity grows with the number of alerts raised. These limitations result in an overall imperfect intrusion detection in the infrastructure. 

We propose to alleviate this assumption for a real life use case through the optimization of the infrastructure monitoring. We formulate a Resource Allocation (RA) problem, in which a defender will have to select a limited number of network nodes for monitoring the infrastructure in order to detect attacks against the migration process. We aim at determining which nodes in the infrastructure will provide the best coverage to detect these attacks. 
In this chapter we investigate the use of Markov Decision Processes (MDP) to solve this RA problem.

% Two formalisms for the resource allocation problem have been reviewed in Chapter~\ref{sec:sota}.
% The Game Theory formalism is used to represent both attacker and defender and how one can attack (resp. defend) the infrastructure.
% This formalism provides different types of games, each defining different interactions between the attacker and the defender.
% \FC{Comment expliquer qu'on fait une review de la th\'eorie des jeux mais que l'on en fait rien apr\`e rien apr\`es ?}
% We investigate the feasibility of using Game Theory to model different attacks on the infrastructure and similarly different defense strategies to protect it.
% Game Theory does not explore the resource allocation problem in an SDN environment, and we expect some applicability in our study.

