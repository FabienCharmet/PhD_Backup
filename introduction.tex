\subsection{Motivation}

The emergence of online services, and the cost-appealing solution to outsource the exploitation of computer infrastructures has led to a drastic increase in the needs for individual resources allocated to users. Decades ago, a simple observation made that having one physical equipment per user need was clearly impossible. Additionally, only a fraction of the power of each computer was actually used and a lot remained to be exploited.
Exploiting those idle resources became a crucial research topic, that would pave the way to resource virtualization. There are three main resources that have been virtualized over the past decades are namely compute, storage and memory.
These resources typically represent the basic needs for a computer.
Virtualization gained a lot of popularity, and led to the development ouf Cloud Computing, where computer infrastructure, business information and software applications would not be handle on site anymore but deployed instead in an external site, \ie the Cloud.

Among all the resources that can be virtualized inside a computer system, the network resource have not benefited from the same performance in virtualization.
There are several primitives that have existed for a long time now, such as VLAN, VRF or MPLS tags, but the heterogeneity of network equipments and the complexity in maintaining coherent network configurations on a large scale did not provide a profitable ground for these primitives to be turned into efficient network virtualization solutions.

However, everything changed with the rise of Software Defined Networking


\subsection{Evaluating the security of the migration process}

\subsection{Research Issues}

\subsection{Contributions}
We present in this thesis \FC{INSERT NAME HERE} 


\subsection{Organization}